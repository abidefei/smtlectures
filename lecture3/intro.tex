\begin{frame}
  \frametitle{Disclaimer}

  The SAT-Solving algorithm described as follows is {\bf not} the
  only approach existing
  \begin{itemize}
    \item BDDs
    \item Quantifier Elimination
    \item Random SAT-Solving
  \end{itemize}
  \vfill
  We study the DPLL/CDCL approach as it is precise (not approximate),
  it uses a linear amount of memory, and very robust
  \vfill
  It is fair to say that it is the most used approach by industry
  for pure solving
  \vfill
  The approach evolved in many years, many groups have contributed. Here
  we do not see the history of this evolution but just the {\bf final product} 

\end{frame}

\begin{frame}
  \frametitle{Complexity Considerations}

  SAT is ``the'' NP-Complete problem
  \vfill
  It is unlikely to be solved in polynomial time
  \vfill 
  Most likely, it takes some $O(2^n)$ time complexity
  for an algorithm to solve SAT
  \vfill 
  Complexity of SAT cannot be alleviated by faster
  machines (only by parallelism, if we ever reach
  that technology). Suppose you can execute $M$ instructions
  in 1 hour, then the maximum $n$ you can handle is
  \vfill
  \begin{center}
  \begin{tabular}{ll}
    \hline
    Speed of machine & Max input size in 1 hour \\
    \hline
    1x               & $log_2(M) = n$ \\
    100x             & $log_2(100 \cdot M) \approx n + 3.3$ \\
    1000000x         & $log_2(1000000 \cdot M) \approx n + 19$ \\
    \hline
  \end{tabular}
  \end{center}

\end{frame}

\begin{frame}
  \frametitle{CNF Formul\ae}

  \scriptsize

  From now on we shall focus on solving \formulae in Conjunctive Normal Form {\bf CNF}

  $$ C_1 \wedge C_2  \wedge \ldots \wedge C_n  $$  

  where each $C_i$ is a {\bf clause}, a disjunction of the kind

  $$( l_1 \vee l_2 \vee \ldots l_m )$$  

  where every  $l_i$ is a {\bf literal}, which is a variable or a negated Boolean variable

  $$
  a\ \mbox{ or }\ \neg a
  $$

  \vfill
  \pause
  For simplicity we will write them as a set of clauses, omitting the $\wedge$, like

  $$
  \begin{array}{l}
  (\neg a_1 \vee a_2) \\
  (\neg a_1 \vee a_3 \vee a_9) \\
  (\neg a_2 \vee \neg a_3 \vee a_4) \\
  (\neg a_4 \vee a_5 \vee a_{10})
  \end{array}
  $$

\end{frame}

\begin{frame}
  \frametitle{Basic Notation}

  We indicate a (possibly partial) {\bf assignment} as the set of literals that
  are $\top$ under it, i.e., we write the assignment 
  $$
  \{ a_1 \mapsto \top, a_2 \mapsto \top, a_3 \mapsto \bot, a_4 \mapsto \bot, a_9 \mapsto \top \}
  $$
  as
  $$
  \{ a_1, a_2, \neg a_3, \neg a_4, a_9 \}
  $$
  \vfill
  \pause
  An assignment is used to evaluate a formula, e.g.,
  $$
  \begin{array}{l}
  (\colone{\neg a_1} \vee \coltwo{a_2}) \\
  (\colone{\neg a_1} \vee \colone{a_3} \vee \coltwo{a_9}) \\
  (\colone{\neg a_2} \vee \coltwo{\neg a_3} \vee \colone{a_4}) \\
  (\coltwo{\neg a_4} \vee a_5 \vee a_{10})
  \end{array}
  $$
  evaluates to $\top$ under the assignment above 
  \pause
  \begin{itemize}
    \item at least one \coltwo{this colored} literal in each clause to make it $\top$
    \item all \colone{this colored} literals in one clause to make it $\bot$
  \end{itemize}

\end{frame}

\begin{frame}
  \frametitle{Simple Facts}

  SAT-Solving is (the art of) finding the assignment satisfying all clauses
  \vfill
  \pause
  The assignment evolves {\bf incrementally} by taking {\bf Decisions}, 
  starting from empty $\{\ \}$, but it can be {\bf backtracked} if found wrong
  \vfill
  \pause
  The evolution of the assignment can be represented as a {\bf tree}
  \vfill
  \pause
  \begin{tabular}{ccc}
    \begin{minipage}{.4\textwidth}
    $$
      \begin{array}{l}
      (\coltwoat{\neg a_1}{5-7} \vee a_3 \vee a_9) \\
      (\coloneat{\coltwoat{\neg a_2}{6}}{7} \vee \neg a_3 \vee a_4) \\
      (\coloneat{a_1}{5-7} \vee \coltwoat{\coloneat{a_2}{6}}{7}) \\
      (\neg a_4 \vee a_5 \vee a_{10})
      \end{array}
    $$
    \begin{overlayarea}{\textwidth}{.3cm}
      \only<4|handout:0>{$\{\ \}$}
      \only<5|handout:0>{$\{ \neg a_1 \}$}
      \only<6|handout:0>{$\{ \neg a_1, \neg a_2 \}$}
      \only<7|handout:0>{$\{ \neg a_1, a_2 \}$}
      \only<8>{$\{ \ldots \}$}
    \end{overlayarea}
    \end{minipage}
    & ~~~~ &
    \begin{minipage}{.4\textwidth}
    \begin{overlayarea}{\textwidth}{5cm}
      \only<1-3|handout:0>{\scalebox{.7}{\input{search_1.pdf_t}}}
      \only<4|handout:0>{\scalebox{.7}{\input{search_1_1.pdf_t}}}
      \only<5|handout:0>{\scalebox{.7}{\input{search_1_2.pdf_t}}}
      \only<6|handout:0>{\scalebox{.7}{\input{search_1_3.pdf_t}}}
      \only<7|handout:0>{\scalebox{.7}{\input{search_1_4.pdf_t}}}
      \only<8>{\scalebox{.7}{\input{search_1_5.pdf_t}}}
    \end{overlayarea}
    \end{minipage}
  \end{tabular}

\end{frame}

\begin{frame}
  \frametitle{Simple Facts}

  There are assignments which can be trivially driven towards the right direction.
  In the example below, given the current assignment, the third clause can be satisfied
  only by setting $a_2 \mapsto \top$ 
  \vfill
  \begin{tabular}{ccc}
    \begin{minipage}{.4\textwidth}
    $$
      \begin{array}{l}
      (\coltwoat{\neg a_1}{1-} \vee a_3 \vee a_9) \\
      (\coloneat{\neg a_2}{2-} \vee \neg a_3 \vee a_4) \\
      (\coloneat{a_1}{1-} \vee \coltwoat{a_2}{2-}) \\
      (\neg a_4 \vee a_5 \vee a_{10})
      \end{array}
    $$
    \begin{overlayarea}{\textwidth}{.3cm}
      \only<1|handout:0>{$\{ \neg a_1 \}$}
      \only<2->{$\{ \neg a_1, a_2 \}$}
    \end{overlayarea}
    \end{minipage}
    & ~~~~ &
    \begin{minipage}{.4\textwidth}
    \begin{overlayarea}{\textwidth}{4cm}
      \only<1:handout0>{\scalebox{.7}{\input{search_2_1.pdf_t}}}
      \only<2->{\scalebox{.7}{\input{search_2_2.pdf_t}}}
    \end{overlayarea}
    \end{minipage}
  \end{tabular}
  \vfill
  \pause
  \pause
  This step is called {\bf Boolean Constraint Propagation} (BCP). It represents
  a {\bf forced} deduction. It triggers whenever all literals but one are assigned $\bot$
  $$
  (\colone{\neg a_1} \vee \colone{a_2} \vee \colone{\neg a_3} \vee \colone{a_4} \vee \neg a_5)
  $$

\end{frame}

\begin{frame}[fragile]
  \frametitle{Simple Facts}

  {\bf Decision-level}: in an assignment, it is the number of decisions taken 
  \vfill
  Clearly, BCPs do not contribute to increase the decision level
  \vfill
  When necessary, we will indicate decision level on top of literals $\dec{a}{5}$ in the assignment 
  \vfill
  \pause
  Example:
  \begin{center}
  \begin{tabular}{ccc}
    \begin{minipage}{.4\textwidth}
      \vfill
      $\{ \dec{a_{10}}{0}, \dec{\neg a_{1}}{1}, \dec{a_3}{2}, \dec{\neg a_4}{3}, \dec{a_5}{3} \}$
      \vfill
    \end{minipage}
    & &
    \begin{minipage}{.4\textwidth}
      \scalebox{.4}{\input{search_4.pdf_t}}
    \end{minipage}
  \end{tabular}
  \end{center}
  
\end{frame}

\begin{frame}
  \frametitle{Simple Facts}
  
  The process of finding the satisfying assignment is called {\bf search}, and
  in its basic version it evolves with {\bf Decisions}, {\bf BCPs}, and
  {\bf backtracking}

  \begin{center}
  \scalebox{.5}{\input{search_3.pdf_t}}
  \end{center}

\end{frame}
