\subsection{Clause Learning}

\begin{frame}
  \frametitle{From DPLL to CDCL}

  \scriptsize

  Consider the following scenario before and after BCP
  \vfill
  \begin{tabular}{ccc}
  \begin{minipage}{.4\textwidth}
  $$
  \begin{array}{l}
    \ldots \\
    (\colone{\neg a_{10}} \vee \colone{\neg a_1} \vee a_4) \\
    (\colone{a_3} \vee \colone{\neg a_1} \vee a_5) \\
    (\neg a_4 \vee a_6) \\
    (\neg a_5 \vee \neg a_6) \\
    \ldots \\
    \\
    \{ \dec{a_{10}}{0}, \dec{\neg a_3}{1}, \dec{a_7}{2}, \dec{\neg a_2}{3}, \dec{a_1}{4} \}
  \end{array}
  $$
  \end{minipage}
  & ~~~~ &
  \begin{minipage}{.5\textwidth}
  $$
  \begin{array}{l}
    \ldots \\
    (\colone{\neg a_{10}} \vee \colone{\neg a_1} \vee \coltwo{a_4}) \\
    (\colone{a_3} \vee \colone{\neg a_1} \vee \coltwo{a_5}) \\
    (\colone{\neg a_4} \vee \coltwo{a_6}) \\
    (\colone{\neg a_5} \vee \colone{\neg a_6}) \\
    \ldots \\
    \\
    \{ \dec{a_{10}}{0}, \dec{\neg a_3}{1}, \dec{a_7}{2}, \dec{\neg a_2}{3}, \dec{a_1}{4}, \dec{a_4}{4}, \dec{a_5}{4}, \dec{a_6}{4}\}
  \end{array}
  $$
  \end{minipage}
  \end{tabular}
  \vfill
  \pause
  BCP leads to a conflict, but, what is the {\bf reason} for it ?
  Do $a_7$ and $a_2$ play any role ?
  \vfill
  \pause
  Does it make sense to consider the assignments (which backtracking would produce) ?\\
  $\{ \dec{a_{10}}{0}, \dec{\neg a_3}{1}, \dec{\neg a_7}{2}, \dec{\neg a_2}{3}, \dec{a_1}{4} \}$ ---
  $\{ \dec{a_{10}}{0}, \dec{\neg a_3}{1}, \dec{a_7}{2}, \dec{\neg a_2}{3}, \dec{a_1}{4} \}$ ---
  $\{ \dec{a_{10}}{0}, \dec{\neg a_3}{1}, \dec{a_7}{2}, \dec{a_2}{3}, \dec{a_1}{4} \}$ \\
  \vfill
  \pause
  No, because {\em whenever $a_{10}$, $\neg a_3$ is assigned, then $a_1$ must not be set to $\top$}. \pause
  This translates to an additional clause, which can be {\em learnt}, i.e., it can be added to the formula 
  $$(\neg a_{10} \vee a_3 \vee \neg a_1)$$
   
\end{frame}

\begin{frame}
  \frametitle{From DPLL to CDCL}

  Search
  \vfill

  \begin{tabular}{ccc}
    Without learnt clause & & With learnt clause \\ \\
    \begin{minipage}{.4\textwidth}
      \scalebox{.3}{\input{search_5.pdf_t}}
    \end{minipage}
    & ~~~~ & 
    \begin{minipage}{.4\textwidth}
      \scalebox{.3}{\input{search_6.pdf_t}}
    \end{minipage}
  \end{tabular}

\end{frame}

\subsection{Conflict Analysis}

\begin{frame}
  \frametitle{Deriving the clause to be learnt}

  \scriptsize

  The clause to be learnt is derived from the conflict ``caused'' by BCP
  (conflict is never caused by a Decision)
  \vfill
  \pause
  To better {\bf analyze} a conflict we need to look at the {\bf implication graph}
  \vfill
  \begin{tabular}{ccc}
  \begin{minipage}{.4\textwidth}
  $$
  \begin{array}{l}
    \ldots \\
    (\colone{\neg a_{10}} \vee \colone{\neg a_1} \vee \coltwo{a_4}) \\
    (\colone{a_3} \vee \colone{\neg a_1} \vee \coltwo{a_5}) \\
    (\colone{\neg a_4} \vee \coltwo{a_6}) \\
    (\colone{\neg a_5} \vee \colone{\neg a_6}) \\
    \ldots \\  \\
    \{ \dec{a_{10}}{0}, \dec{\neg a_3}{1}, \dec{a_7}{2}, \dec{\neg a_2}{3}, \dec{a_1}{4}, \dec{a_4}{4}, \dec{a_5}{4}, \dec{a_6}{4}\}
  \end{array}
  $$
  \end{minipage}
  & ~~~ &
  \begin{minipage}{.4\textwidth}
  \begin{overlayarea}{.4\textwidth}{4cm}
    \only<2|handout:0>{\scalebox{.4}{\input{impl_graph_1.pdf_t}}}
    \only<3|handout:0>{\scalebox{.4}{\input{impl_graph_2.pdf_t}}}
    \only<4|handout:0>{\scalebox{.4}{\input{impl_graph_3.pdf_t}}}
    \only<5->{\scalebox{.4}{\input{impl_graph_4.pdf_t}}}
  \end{overlayarea}
  \end{minipage}
  \end{tabular}
  \vfill
  \pause\pause
  Any cut of the graph that separates the conflict from the decision variables
  represents a possible learnt clause
  \vfill
  \pause
  In this course we take the clause that contains 
  {\bf only one variable of the current decision level}

\end{frame}

\begin{frame}
  \frametitle{Deriving the clause to be learnt}

  \scriptsize

  In practice, we use {\bf resolution} steps to compute the learnt clause:
  \begin{itemize}
    \item we start from the clause with all literals to $\bot$ (conflicting clause)
    \item iteratively, we take the clause that propagated the last literal on the trail
	  and we apply resolution
    \item we stop when only one literal from the current decision level is left in the clause 
  \end{itemize}
  \vfill
  \pause
  \begin{tabular}{ccc}
    \begin{minipage}{.4\textwidth}
      \begin{overlayarea}{.4\textwidth}{3cm}
      \only<2|handout:0>{\scalebox{.4}{\input{ca_1.pdf_t}}}
      \only<3|handout:0>{\scalebox{.4}{\input{ca_2.pdf_t}}}
      \only<4|handout:0>{\scalebox{.4}{\input{ca_3.pdf_t}}}
      \only<5|handout:0>{\scalebox{.4}{\input{ca_4.pdf_t}}}
      \only<6|handout:0>{\scalebox{.4}{\input{ca_5.pdf_t}}}
      \only<7|handout:0>{\scalebox{.4}{\input{ca_6.pdf_t}}}
      \only<8->{\scalebox{.4}{\input{ca_7.pdf_t}}}
      \end{overlayarea}
    \end{minipage}
    & ~~~~~~~~~~~~~~~~~~~~~~~~~~~~~~ &
    \begin{minipage}{.4\textwidth}
      \begin{tabular}{ccl}
	\hline
	Trail & dl & Reason \\
	\hline
	$a_{10}$   & 0 & $( a_{10} )$ \\
	$\neg a_3$ & 1 & Decision \\
	$a_7$      & 2 & Decision \\
	$\neg a_2$ & 3 & Decision \\
	$a_1$      & 4 & Decision \\
	$a_4$      & 4 & $(\neg a_{10} \vee \neg a_1 \vee a_4)$ \\
	$a_5$      & 4 & $(a_3 \vee \neg a_1 \vee a_5)$ \\
	$a_6$      & 4 & $(\neg a_4 \vee a_6)$ \\
	\hline
      \end{tabular}
      \medskip \\
      $\{ \dec{a_{10}}{0}, \dec{\neg a_3}{1}, \dec{a_7}{2}, \dec{\neg a_2}{3}, \dec{a_1}{4}, \dec{a_4}{4}, \dec{a_5}{4}, \dec{a_6}{4}\}$
    \end{minipage}
  \end{tabular}
  \vfill
  \onslide<9>{We say that $(\neg a_{10} \vee a_3 \vee \neg a_1)$ is the {\bf conflict clause} and it is the one we learn}

\end{frame}

\subsection{Non-Chronological Bactracking}

\begin{frame}
  \frametitle{Non-chronological backtracking}

  So far we have been backtracking {\bf chronologically}, but
  \vfill
  given our (wrong) assignment $\{ \dec{a_{10}}{0}, \dec{\neg a_3}{1}, \dec{a_7}{2}, \dec{\neg a_2}{3}, \dec{a_1}{4}, \dec{a_4}{4}, \dec{a_5}{4}, \dec{a_6}{4}\}$
  and the computed conflict clause $(\neg a_{10} \vee a_3 \vee \neg a_1)$, is it really clever to backtrack to level 3 ?
  \vfill
  \pause
  The correct level to backtrack to is: 
  \begin{center}
  ``the level that would have propagated $\neg a_1$ if we had the clause
  $(\neg a_{10} \vee a_3 \vee \neg a_1)$ as part of the original formula''
  \end{center}
  \vfill
  \pause
  In our case, it is level 1, because assignment $\{ \dec{a_{10}}{0}, \dec{\neg a_3}{1} \}$ is
  sufficient to propagate $\neg a_1$ in $(\colone{\neg a_{10}} \vee \colone{a_3} \vee \neg a_1)$
  \vfill
  \pause
  We say that $\neg a_1$ is the {\bf asserting literal} as it becomes true by BCP once 
  we have backtracked to the correct decision level 
\end{frame}

\begin{frame}
  \frametitle{CDCL: Conflict Driven Clause Learning}

  Search
  \vfill

  \begin{tabular}{ccc}
    DPLL                        & & CDCL  \\ 
    no learning                 & & conflict-driven learning \\
    chronological backtracking  & & non-chronological backtracking \\ \\
    \begin{minipage}{.4\textwidth}
      \scalebox{.3}{\input{search_7.pdf_t}}
    \end{minipage}
    & ~~~~ & 
    \begin{minipage}{.4\textwidth}
      \scalebox{.3}{\input{search_8.pdf_t}}
    \end{minipage}
  \end{tabular}

\end{frame}

\begin{frame}
  \frametitle{The CDCL Procedure}

  \scriptsize

  \vfill

  \begin{tabbing}
  as \= a \= a \= a \= as \= asdfsdfasdfasdfasdfasdfasdfasdfasdfasdf \= \kill
  \> $dl = 0$; $trail = \{\ \}$;                              \> \> \> \> \> // Decision level, assignment\\ \\
  \> while ( $true$ ) \\ \\
  \> \> if ( BCP( ) == conflict )                                \> \> \> \> // Do BCP until possible \\
  \> \> \> if ( $dl == 0$ ) return $unsat$;                         \> \> \> // Unresolvable conflict \\
  \> \> \> $C,dl =$ {\sc AnalyzeConflict}( );                           \> \> \> // Compute conf. clause, and dec. level \\
  \> \> \> {\sc AddClause}( $C$ );                                      \> \> \> // Add $C$ to clause database \\
  \> \> \> {\sc BacktrackTo}( $dl$ );                                   \> \> \> // Backtracking (shrinks $trail$) \\
  \> \> else \\
  \> \> \> if ( ``all variables assigned'' ) return $sat$;          \> \> \> // $trail$ holds satisfying assignment \\
  \> \> \> $l = $ {\sc Decision}( );                                      \> \> \> // Do another decision \\
  \> \> \> $trail = trail \cup \{ l \}$ \\                          
  \> \> \> $dl = dl + 1$;                                           \> \> \> // Increase decision level \\
  \end{tabbing}

  \vfill

\end{frame}

\begin{frame}
  \frametitle{The CDCL Procedure}

  \begin{center}
  \scalebox{.45}{\subsection{Clause Learning}

\begin{frame}
  \frametitle{From DPLL to CDCL}

  \scriptsize

  Consider the following scenario before and after BCP
  \vfill
  \begin{tabular}{ccc}
  \begin{minipage}{.4\textwidth}
  $$
  \begin{array}{l}
    \ldots \\
    (\colone{\neg a_{10}} \vee \colone{\neg a_1} \vee a_4) \\
    (\colone{a_3} \vee \colone{\neg a_1} \vee a_5) \\
    (\neg a_4 \vee a_6) \\
    (\neg a_5 \vee \neg a_6) \\
    \ldots \\
    \\
    \{ \dec{a_{10}}{0}, \dec{\neg a_3}{1}, \dec{a_7}{2}, \dec{\neg a_2}{3}, \dec{a_1}{4} \}
  \end{array}
  $$
  \end{minipage}
  & ~~~~ &
  \begin{minipage}{.5\textwidth}
  $$
  \begin{array}{l}
    \ldots \\
    (\colone{\neg a_{10}} \vee \colone{\neg a_1} \vee \coltwo{a_4}) \\
    (\colone{a_3} \vee \colone{\neg a_1} \vee \coltwo{a_5}) \\
    (\colone{\neg a_4} \vee \coltwo{a_6}) \\
    (\colone{\neg a_5} \vee \colone{\neg a_6}) \\
    \ldots \\
    \\
    \{ \dec{a_{10}}{0}, \dec{\neg a_3}{1}, \dec{a_7}{2}, \dec{\neg a_2}{3}, \dec{a_1}{4}, \dec{a_4}{4}, \dec{a_5}{4}, \dec{a_6}{4}\}
  \end{array}
  $$
  \end{minipage}
  \end{tabular}
  \vfill
  \pause
  BCP leads to a conflict, but, what is the {\bf reason} for it ?
  Do $a_7$ and $a_2$ play any role ?
  \vfill
  \pause
  Does it make sense to consider the assignments (which backtracking would produce) ?\\
  $\{ \dec{a_{10}}{0}, \dec{\neg a_3}{1}, \dec{\neg a_7}{2}, \dec{\neg a_2}{3}, \dec{a_1}{4} \}$ ---
  $\{ \dec{a_{10}}{0}, \dec{\neg a_3}{1}, \dec{a_7}{2}, \dec{\neg a_2}{3}, \dec{a_1}{4} \}$ ---
  $\{ \dec{a_{10}}{0}, \dec{\neg a_3}{1}, \dec{a_7}{2}, \dec{a_2}{3}, \dec{a_1}{4} \}$ \\
  \vfill
  \pause
  No, because {\em whenever $a_{10}$, $\neg a_3$ is assigned, then $a_1$ must not be set to $\top$}. \pause
  This translates to an additional clause, which can be {\em learnt}, i.e., it can be added to the formula 
  $$(\neg a_{10} \vee a_3 \vee \neg a_1)$$
   
\end{frame}

\begin{frame}
  \frametitle{From DPLL to CDCL}

  Search
  \vfill

  \begin{tabular}{ccc}
    Without learnt clause & & With learnt clause \\ \\
    \begin{minipage}{.4\textwidth}
      \scalebox{.3}{\input{search_5.pdf_t}}
    \end{minipage}
    & ~~~~ & 
    \begin{minipage}{.4\textwidth}
      \scalebox{.3}{\input{search_6.pdf_t}}
    \end{minipage}
  \end{tabular}

\end{frame}

\subsection{Conflict Analysis}

\begin{frame}
  \frametitle{Deriving the clause to be learnt}

  \scriptsize

  The clause to be learnt is derived from the conflict ``caused'' by BCP
  (conflict is never caused by a Decision)
  \vfill
  \pause
  To better {\bf analyze} a conflict we need to look at the {\bf implication graph}
  \vfill
  \begin{tabular}{ccc}
  \begin{minipage}{.4\textwidth}
  $$
  \begin{array}{l}
    \ldots \\
    (\colone{\neg a_{10}} \vee \colone{\neg a_1} \vee \coltwo{a_4}) \\
    (\colone{a_3} \vee \colone{\neg a_1} \vee \coltwo{a_5}) \\
    (\colone{\neg a_4} \vee \coltwo{a_6}) \\
    (\colone{\neg a_5} \vee \colone{\neg a_6}) \\
    \ldots \\  \\
    \{ \dec{a_{10}}{0}, \dec{\neg a_3}{1}, \dec{a_7}{2}, \dec{\neg a_2}{3}, \dec{a_1}{4}, \dec{a_4}{4}, \dec{a_5}{4}, \dec{a_6}{4}\}
  \end{array}
  $$
  \end{minipage}
  & ~~~ &
  \begin{minipage}{.4\textwidth}
  \begin{overlayarea}{.4\textwidth}{4cm}
    \only<2|handout:0>{\scalebox{.4}{\input{impl_graph_1.pdf_t}}}
    \only<3|handout:0>{\scalebox{.4}{\input{impl_graph_2.pdf_t}}}
    \only<4|handout:0>{\scalebox{.4}{\input{impl_graph_3.pdf_t}}}
    \only<5->{\scalebox{.4}{\input{impl_graph_4.pdf_t}}}
  \end{overlayarea}
  \end{minipage}
  \end{tabular}
  \vfill
  \pause\pause
  Any cut of the graph that separates the conflict from the decision variables
  represents a possible learnt clause
  \vfill
  \pause
  In this course we take the clause that contains 
  {\bf only one variable of the current decision level}

\end{frame}

\begin{frame}
  \frametitle{Deriving the clause to be learnt}

  \scriptsize

  In practice, we use {\bf resolution} steps to compute the learnt clause:
  \begin{itemize}
    \item we start from the clause with all literals to $\bot$ (conflicting clause)
    \item iteratively, we take the clause that propagated the last literal on the trail
	  and we apply resolution
    \item we stop when only one literal from the current decision level is left in the clause 
  \end{itemize}
  \vfill
  \pause
  \begin{tabular}{ccc}
    \begin{minipage}{.4\textwidth}
      \begin{overlayarea}{.4\textwidth}{3cm}
      \only<2|handout:0>{\scalebox{.4}{\input{ca_1.pdf_t}}}
      \only<3|handout:0>{\scalebox{.4}{\input{ca_2.pdf_t}}}
      \only<4|handout:0>{\scalebox{.4}{\input{ca_3.pdf_t}}}
      \only<5|handout:0>{\scalebox{.4}{\input{ca_4.pdf_t}}}
      \only<6|handout:0>{\scalebox{.4}{\input{ca_5.pdf_t}}}
      \only<7|handout:0>{\scalebox{.4}{\input{ca_6.pdf_t}}}
      \only<8->{\scalebox{.4}{\input{ca_7.pdf_t}}}
      \end{overlayarea}
    \end{minipage}
    & ~~~~~~~~~~~~~~~~~~~~~~~~~~~~~~ &
    \begin{minipage}{.4\textwidth}
      \begin{tabular}{ccl}
	\hline
	Trail & dl & Reason \\
	\hline
	$a_{10}$   & 0 & $( a_{10} )$ \\
	$\neg a_3$ & 1 & Decision \\
	$a_7$      & 2 & Decision \\
	$\neg a_2$ & 3 & Decision \\
	$a_1$      & 4 & Decision \\
	$a_4$      & 4 & $(\neg a_{10} \vee \neg a_1 \vee a_4)$ \\
	$a_5$      & 4 & $(a_3 \vee \neg a_1 \vee a_5)$ \\
	$a_6$      & 4 & $(\neg a_4 \vee a_6)$ \\
	\hline
      \end{tabular}
      \medskip \\
      $\{ \dec{a_{10}}{0}, \dec{\neg a_3}{1}, \dec{a_7}{2}, \dec{\neg a_2}{3}, \dec{a_1}{4}, \dec{a_4}{4}, \dec{a_5}{4}, \dec{a_6}{4}\}$
    \end{minipage}
  \end{tabular}
  \vfill
  \onslide<9>{We say that $(\neg a_{10} \vee a_3 \vee \neg a_1)$ is the {\bf conflict clause} and it is the one we learn}

\end{frame}

\subsection{Non-Chronological Bactracking}

\begin{frame}
  \frametitle{Non-chronological backtracking}

  So far we have been backtracking {\bf chronologically}, but
  \vfill
  given our (wrong) assignment $\{ \dec{a_{10}}{0}, \dec{\neg a_3}{1}, \dec{a_7}{2}, \dec{\neg a_2}{3}, \dec{a_1}{4}, \dec{a_4}{4}, \dec{a_5}{4}, \dec{a_6}{4}\}$
  and the computed conflict clause $(\neg a_{10} \vee a_3 \vee \neg a_1)$, is it really clever to backtrack to level 3 ?
  \vfill
  \pause
  The correct level to backtrack to is: 
  \begin{center}
  ``the level that would have propagated $\neg a_1$ if we had the clause
  $(\neg a_{10} \vee a_3 \vee \neg a_1)$ as part of the original formula''
  \end{center}
  \vfill
  \pause
  In our case, it is level 1, because assignment $\{ \dec{a_{10}}{0}, \dec{\neg a_3}{1} \}$ is
  sufficient to propagate $\neg a_1$ in $(\colone{\neg a_{10}} \vee \colone{a_3} \vee \neg a_1)$
  \vfill
  \pause
  We say that $\neg a_1$ is the {\bf asserting literal} as it becomes true by BCP once 
  we have backtracked to the correct decision level 
\end{frame}

\begin{frame}
  \frametitle{CDCL: Conflict Driven Clause Learning}

  Search
  \vfill

  \begin{tabular}{ccc}
    DPLL                        & & CDCL  \\ 
    no learning                 & & conflict-driven learning \\
    chronological backtracking  & & non-chronological backtracking \\ \\
    \begin{minipage}{.4\textwidth}
      \scalebox{.3}{\input{search_7.pdf_t}}
    \end{minipage}
    & ~~~~ & 
    \begin{minipage}{.4\textwidth}
      \scalebox{.3}{\input{search_8.pdf_t}}
    \end{minipage}
  \end{tabular}

\end{frame}

\begin{frame}
  \frametitle{The CDCL Procedure}

  \scriptsize

  \vfill

  \begin{tabbing}
  as \= a \= a \= a \= as \= asdfsdfasdfasdfasdfasdfasdfasdfasdfasdf \= \kill
  \> $dl = 0$; $trail = \{\ \}$;                              \> \> \> \> \> // Decision level, assignment\\ \\
  \> while ( $true$ ) \\ \\
  \> \> if ( BCP( ) == conflict )                                \> \> \> \> // Do BCP until possible \\
  \> \> \> if ( $dl == 0$ ) return $unsat$;                         \> \> \> // Unresolvable conflict \\
  \> \> \> $C,dl =$ {\sc AnalyzeConflict}( );                           \> \> \> // Compute conf. clause, and dec. level \\
  \> \> \> {\sc AddClause}( $C$ );                                      \> \> \> // Add $C$ to clause database \\
  \> \> \> {\sc BacktrackTo}( $dl$ );                                   \> \> \> // Backtracking (shrinks $trail$) \\
  \> \> else \\
  \> \> \> if ( ``all variables assigned'' ) return $sat$;          \> \> \> // $trail$ holds satisfying assignment \\
  \> \> \> $l = $ {\sc Decision}( );                                      \> \> \> // Do another decision \\
  \> \> \> $trail = trail \cup \{ l \}$ \\                          
  \> \> \> $dl = dl + 1$;                                           \> \> \> // Increase decision level \\
  \end{tabbing}

  \vfill

\end{frame}

\begin{frame}
  \frametitle{The CDCL Procedure}

  \begin{center}
  \scalebox{.45}{\subsection{Clause Learning}

\begin{frame}
  \frametitle{From DPLL to CDCL}

  \scriptsize

  Consider the following scenario before and after BCP
  \vfill
  \begin{tabular}{ccc}
  \begin{minipage}{.4\textwidth}
  $$
  \begin{array}{l}
    \ldots \\
    (\colone{\neg a_{10}} \vee \colone{\neg a_1} \vee a_4) \\
    (\colone{a_3} \vee \colone{\neg a_1} \vee a_5) \\
    (\neg a_4 \vee a_6) \\
    (\neg a_5 \vee \neg a_6) \\
    \ldots \\
    \\
    \{ \dec{a_{10}}{0}, \dec{\neg a_3}{1}, \dec{a_7}{2}, \dec{\neg a_2}{3}, \dec{a_1}{4} \}
  \end{array}
  $$
  \end{minipage}
  & ~~~~ &
  \begin{minipage}{.5\textwidth}
  $$
  \begin{array}{l}
    \ldots \\
    (\colone{\neg a_{10}} \vee \colone{\neg a_1} \vee \coltwo{a_4}) \\
    (\colone{a_3} \vee \colone{\neg a_1} \vee \coltwo{a_5}) \\
    (\colone{\neg a_4} \vee \coltwo{a_6}) \\
    (\colone{\neg a_5} \vee \colone{\neg a_6}) \\
    \ldots \\
    \\
    \{ \dec{a_{10}}{0}, \dec{\neg a_3}{1}, \dec{a_7}{2}, \dec{\neg a_2}{3}, \dec{a_1}{4}, \dec{a_4}{4}, \dec{a_5}{4}, \dec{a_6}{4}\}
  \end{array}
  $$
  \end{minipage}
  \end{tabular}
  \vfill
  \pause
  BCP leads to a conflict, but, what is the {\bf reason} for it ?
  Do $a_7$ and $a_2$ play any role ?
  \vfill
  \pause
  Does it make sense to consider the assignments (which backtracking would produce) ?\\
  $\{ \dec{a_{10}}{0}, \dec{\neg a_3}{1}, \dec{\neg a_7}{2}, \dec{\neg a_2}{3}, \dec{a_1}{4} \}$ ---
  $\{ \dec{a_{10}}{0}, \dec{\neg a_3}{1}, \dec{a_7}{2}, \dec{\neg a_2}{3}, \dec{a_1}{4} \}$ ---
  $\{ \dec{a_{10}}{0}, \dec{\neg a_3}{1}, \dec{a_7}{2}, \dec{a_2}{3}, \dec{a_1}{4} \}$ \\
  \vfill
  \pause
  No, because {\em whenever $a_{10}$, $\neg a_3$ is assigned, then $a_1$ must not be set to $\top$}. \pause
  This translates to an additional clause, which can be {\em learnt}, i.e., it can be added to the formula 
  $$(\neg a_{10} \vee a_3 \vee \neg a_1)$$
   
\end{frame}

\begin{frame}
  \frametitle{From DPLL to CDCL}

  Search
  \vfill

  \begin{tabular}{ccc}
    Without learnt clause & & With learnt clause \\ \\
    \begin{minipage}{.4\textwidth}
      \scalebox{.3}{\input{search_5.pdf_t}}
    \end{minipage}
    & ~~~~ & 
    \begin{minipage}{.4\textwidth}
      \scalebox{.3}{\input{search_6.pdf_t}}
    \end{minipage}
  \end{tabular}

\end{frame}

\subsection{Conflict Analysis}

\begin{frame}
  \frametitle{Deriving the clause to be learnt}

  \scriptsize

  The clause to be learnt is derived from the conflict ``caused'' by BCP
  (conflict is never caused by a Decision)
  \vfill
  \pause
  To better {\bf analyze} a conflict we need to look at the {\bf implication graph}
  \vfill
  \begin{tabular}{ccc}
  \begin{minipage}{.4\textwidth}
  $$
  \begin{array}{l}
    \ldots \\
    (\colone{\neg a_{10}} \vee \colone{\neg a_1} \vee \coltwo{a_4}) \\
    (\colone{a_3} \vee \colone{\neg a_1} \vee \coltwo{a_5}) \\
    (\colone{\neg a_4} \vee \coltwo{a_6}) \\
    (\colone{\neg a_5} \vee \colone{\neg a_6}) \\
    \ldots \\  \\
    \{ \dec{a_{10}}{0}, \dec{\neg a_3}{1}, \dec{a_7}{2}, \dec{\neg a_2}{3}, \dec{a_1}{4}, \dec{a_4}{4}, \dec{a_5}{4}, \dec{a_6}{4}\}
  \end{array}
  $$
  \end{minipage}
  & ~~~ &
  \begin{minipage}{.4\textwidth}
  \begin{overlayarea}{.4\textwidth}{4cm}
    \only<2|handout:0>{\scalebox{.4}{\input{impl_graph_1.pdf_t}}}
    \only<3|handout:0>{\scalebox{.4}{\input{impl_graph_2.pdf_t}}}
    \only<4|handout:0>{\scalebox{.4}{\input{impl_graph_3.pdf_t}}}
    \only<5->{\scalebox{.4}{\input{impl_graph_4.pdf_t}}}
  \end{overlayarea}
  \end{minipage}
  \end{tabular}
  \vfill
  \pause\pause
  Any cut of the graph that separates the conflict from the decision variables
  represents a possible learnt clause
  \vfill
  \pause
  In this course we take the clause that contains 
  {\bf only one variable of the current decision level}

\end{frame}

\begin{frame}
  \frametitle{Deriving the clause to be learnt}

  \scriptsize

  In practice, we use {\bf resolution} steps to compute the learnt clause:
  \begin{itemize}
    \item we start from the clause with all literals to $\bot$ (conflicting clause)
    \item iteratively, we take the clause that propagated the last literal on the trail
	  and we apply resolution
    \item we stop when only one literal from the current decision level is left in the clause 
  \end{itemize}
  \vfill
  \pause
  \begin{tabular}{ccc}
    \begin{minipage}{.4\textwidth}
      \begin{overlayarea}{.4\textwidth}{3cm}
      \only<2|handout:0>{\scalebox{.4}{\input{ca_1.pdf_t}}}
      \only<3|handout:0>{\scalebox{.4}{\input{ca_2.pdf_t}}}
      \only<4|handout:0>{\scalebox{.4}{\input{ca_3.pdf_t}}}
      \only<5|handout:0>{\scalebox{.4}{\input{ca_4.pdf_t}}}
      \only<6|handout:0>{\scalebox{.4}{\input{ca_5.pdf_t}}}
      \only<7|handout:0>{\scalebox{.4}{\input{ca_6.pdf_t}}}
      \only<8->{\scalebox{.4}{\input{ca_7.pdf_t}}}
      \end{overlayarea}
    \end{minipage}
    & ~~~~~~~~~~~~~~~~~~~~~~~~~~~~~~ &
    \begin{minipage}{.4\textwidth}
      \begin{tabular}{ccl}
	\hline
	Trail & dl & Reason \\
	\hline
	$a_{10}$   & 0 & $( a_{10} )$ \\
	$\neg a_3$ & 1 & Decision \\
	$a_7$      & 2 & Decision \\
	$\neg a_2$ & 3 & Decision \\
	$a_1$      & 4 & Decision \\
	$a_4$      & 4 & $(\neg a_{10} \vee \neg a_1 \vee a_4)$ \\
	$a_5$      & 4 & $(a_3 \vee \neg a_1 \vee a_5)$ \\
	$a_6$      & 4 & $(\neg a_4 \vee a_6)$ \\
	\hline
      \end{tabular}
      \medskip \\
      $\{ \dec{a_{10}}{0}, \dec{\neg a_3}{1}, \dec{a_7}{2}, \dec{\neg a_2}{3}, \dec{a_1}{4}, \dec{a_4}{4}, \dec{a_5}{4}, \dec{a_6}{4}\}$
    \end{minipage}
  \end{tabular}
  \vfill
  \onslide<9>{We say that $(\neg a_{10} \vee a_3 \vee \neg a_1)$ is the {\bf conflict clause} and it is the one we learn}

\end{frame}

\subsection{Non-Chronological Bactracking}

\begin{frame}
  \frametitle{Non-chronological backtracking}

  So far we have been backtracking {\bf chronologically}, but
  \vfill
  given our (wrong) assignment $\{ \dec{a_{10}}{0}, \dec{\neg a_3}{1}, \dec{a_7}{2}, \dec{\neg a_2}{3}, \dec{a_1}{4}, \dec{a_4}{4}, \dec{a_5}{4}, \dec{a_6}{4}\}$
  and the computed conflict clause $(\neg a_{10} \vee a_3 \vee \neg a_1)$, is it really clever to backtrack to level 3 ?
  \vfill
  \pause
  The correct level to backtrack to is: 
  \begin{center}
  ``the level that would have propagated $\neg a_1$ if we had the clause
  $(\neg a_{10} \vee a_3 \vee \neg a_1)$ as part of the original formula''
  \end{center}
  \vfill
  \pause
  In our case, it is level 1, because assignment $\{ \dec{a_{10}}{0}, \dec{\neg a_3}{1} \}$ is
  sufficient to propagate $\neg a_1$ in $(\colone{\neg a_{10}} \vee \colone{a_3} \vee \neg a_1)$
  \vfill
  \pause
  We say that $\neg a_1$ is the {\bf asserting literal} as it becomes true by BCP once 
  we have backtracked to the correct decision level 
\end{frame}

\begin{frame}
  \frametitle{CDCL: Conflict Driven Clause Learning}

  Search
  \vfill

  \begin{tabular}{ccc}
    DPLL                        & & CDCL  \\ 
    no learning                 & & conflict-driven learning \\
    chronological backtracking  & & non-chronological backtracking \\ \\
    \begin{minipage}{.4\textwidth}
      \scalebox{.3}{\input{search_7.pdf_t}}
    \end{minipage}
    & ~~~~ & 
    \begin{minipage}{.4\textwidth}
      \scalebox{.3}{\input{search_8.pdf_t}}
    \end{minipage}
  \end{tabular}

\end{frame}

\begin{frame}
  \frametitle{The CDCL Procedure}

  \scriptsize

  \vfill

  \begin{tabbing}
  as \= a \= a \= a \= as \= asdfsdfasdfasdfasdfasdfasdfasdfasdfasdf \= \kill
  \> $dl = 0$; $trail = \{\ \}$;                              \> \> \> \> \> // Decision level, assignment\\ \\
  \> while ( $true$ ) \\ \\
  \> \> if ( BCP( ) == conflict )                                \> \> \> \> // Do BCP until possible \\
  \> \> \> if ( $dl == 0$ ) return $unsat$;                         \> \> \> // Unresolvable conflict \\
  \> \> \> $C,dl =$ {\sc AnalyzeConflict}( );                           \> \> \> // Compute conf. clause, and dec. level \\
  \> \> \> {\sc AddClause}( $C$ );                                      \> \> \> // Add $C$ to clause database \\
  \> \> \> {\sc BacktrackTo}( $dl$ );                                   \> \> \> // Backtracking (shrinks $trail$) \\
  \> \> else \\
  \> \> \> if ( ``all variables assigned'' ) return $sat$;          \> \> \> // $trail$ holds satisfying assignment \\
  \> \> \> $l = $ {\sc Decision}( );                                      \> \> \> // Do another decision \\
  \> \> \> $trail = trail \cup \{ l \}$ \\                          
  \> \> \> $dl = dl + 1$;                                           \> \> \> // Increase decision level \\
  \end{tabbing}

  \vfill

\end{frame}

\begin{frame}
  \frametitle{The CDCL Procedure}

  \begin{center}
  \scalebox{.45}{\subsection{Clause Learning}

\begin{frame}
  \frametitle{From DPLL to CDCL}

  \scriptsize

  Consider the following scenario before and after BCP
  \vfill
  \begin{tabular}{ccc}
  \begin{minipage}{.4\textwidth}
  $$
  \begin{array}{l}
    \ldots \\
    (\colone{\neg a_{10}} \vee \colone{\neg a_1} \vee a_4) \\
    (\colone{a_3} \vee \colone{\neg a_1} \vee a_5) \\
    (\neg a_4 \vee a_6) \\
    (\neg a_5 \vee \neg a_6) \\
    \ldots \\
    \\
    \{ \dec{a_{10}}{0}, \dec{\neg a_3}{1}, \dec{a_7}{2}, \dec{\neg a_2}{3}, \dec{a_1}{4} \}
  \end{array}
  $$
  \end{minipage}
  & ~~~~ &
  \begin{minipage}{.5\textwidth}
  $$
  \begin{array}{l}
    \ldots \\
    (\colone{\neg a_{10}} \vee \colone{\neg a_1} \vee \coltwo{a_4}) \\
    (\colone{a_3} \vee \colone{\neg a_1} \vee \coltwo{a_5}) \\
    (\colone{\neg a_4} \vee \coltwo{a_6}) \\
    (\colone{\neg a_5} \vee \colone{\neg a_6}) \\
    \ldots \\
    \\
    \{ \dec{a_{10}}{0}, \dec{\neg a_3}{1}, \dec{a_7}{2}, \dec{\neg a_2}{3}, \dec{a_1}{4}, \dec{a_4}{4}, \dec{a_5}{4}, \dec{a_6}{4}\}
  \end{array}
  $$
  \end{minipage}
  \end{tabular}
  \vfill
  \pause
  BCP leads to a conflict, but, what is the {\bf reason} for it ?
  Do $a_7$ and $a_2$ play any role ?
  \vfill
  \pause
  Does it make sense to consider the assignments (which backtracking would produce) ?\\
  $\{ \dec{a_{10}}{0}, \dec{\neg a_3}{1}, \dec{\neg a_7}{2}, \dec{\neg a_2}{3}, \dec{a_1}{4} \}$ ---
  $\{ \dec{a_{10}}{0}, \dec{\neg a_3}{1}, \dec{a_7}{2}, \dec{\neg a_2}{3}, \dec{a_1}{4} \}$ ---
  $\{ \dec{a_{10}}{0}, \dec{\neg a_3}{1}, \dec{a_7}{2}, \dec{a_2}{3}, \dec{a_1}{4} \}$ \\
  \vfill
  \pause
  No, because {\em whenever $a_{10}$, $\neg a_3$ is assigned, then $a_1$ must not be set to $\top$}. \pause
  This translates to an additional clause, which can be {\em learnt}, i.e., it can be added to the formula 
  $$(\neg a_{10} \vee a_3 \vee \neg a_1)$$
   
\end{frame}

\begin{frame}
  \frametitle{From DPLL to CDCL}

  Search
  \vfill

  \begin{tabular}{ccc}
    Without learnt clause & & With learnt clause \\ \\
    \begin{minipage}{.4\textwidth}
      \scalebox{.3}{\input{search_5.pdf_t}}
    \end{minipage}
    & ~~~~ & 
    \begin{minipage}{.4\textwidth}
      \scalebox{.3}{\input{search_6.pdf_t}}
    \end{minipage}
  \end{tabular}

\end{frame}

\subsection{Conflict Analysis}

\begin{frame}
  \frametitle{Deriving the clause to be learnt}

  \scriptsize

  The clause to be learnt is derived from the conflict ``caused'' by BCP
  (conflict is never caused by a Decision)
  \vfill
  \pause
  To better {\bf analyze} a conflict we need to look at the {\bf implication graph}
  \vfill
  \begin{tabular}{ccc}
  \begin{minipage}{.4\textwidth}
  $$
  \begin{array}{l}
    \ldots \\
    (\colone{\neg a_{10}} \vee \colone{\neg a_1} \vee \coltwo{a_4}) \\
    (\colone{a_3} \vee \colone{\neg a_1} \vee \coltwo{a_5}) \\
    (\colone{\neg a_4} \vee \coltwo{a_6}) \\
    (\colone{\neg a_5} \vee \colone{\neg a_6}) \\
    \ldots \\  \\
    \{ \dec{a_{10}}{0}, \dec{\neg a_3}{1}, \dec{a_7}{2}, \dec{\neg a_2}{3}, \dec{a_1}{4}, \dec{a_4}{4}, \dec{a_5}{4}, \dec{a_6}{4}\}
  \end{array}
  $$
  \end{minipage}
  & ~~~ &
  \begin{minipage}{.4\textwidth}
  \begin{overlayarea}{.4\textwidth}{4cm}
    \only<2|handout:0>{\scalebox{.4}{\input{impl_graph_1.pdf_t}}}
    \only<3|handout:0>{\scalebox{.4}{\input{impl_graph_2.pdf_t}}}
    \only<4|handout:0>{\scalebox{.4}{\input{impl_graph_3.pdf_t}}}
    \only<5->{\scalebox{.4}{\input{impl_graph_4.pdf_t}}}
  \end{overlayarea}
  \end{minipage}
  \end{tabular}
  \vfill
  \pause\pause
  Any cut of the graph that separates the conflict from the decision variables
  represents a possible learnt clause
  \vfill
  \pause
  In this course we take the clause that contains 
  {\bf only one variable of the current decision level}

\end{frame}

\begin{frame}
  \frametitle{Deriving the clause to be learnt}

  \scriptsize

  In practice, we use {\bf resolution} steps to compute the learnt clause:
  \begin{itemize}
    \item we start from the clause with all literals to $\bot$ (conflicting clause)
    \item iteratively, we take the clause that propagated the last literal on the trail
	  and we apply resolution
    \item we stop when only one literal from the current decision level is left in the clause 
  \end{itemize}
  \vfill
  \pause
  \begin{tabular}{ccc}
    \begin{minipage}{.4\textwidth}
      \begin{overlayarea}{.4\textwidth}{3cm}
      \only<2|handout:0>{\scalebox{.4}{\input{ca_1.pdf_t}}}
      \only<3|handout:0>{\scalebox{.4}{\input{ca_2.pdf_t}}}
      \only<4|handout:0>{\scalebox{.4}{\input{ca_3.pdf_t}}}
      \only<5|handout:0>{\scalebox{.4}{\input{ca_4.pdf_t}}}
      \only<6|handout:0>{\scalebox{.4}{\input{ca_5.pdf_t}}}
      \only<7|handout:0>{\scalebox{.4}{\input{ca_6.pdf_t}}}
      \only<8->{\scalebox{.4}{\input{ca_7.pdf_t}}}
      \end{overlayarea}
    \end{minipage}
    & ~~~~~~~~~~~~~~~~~~~~~~~~~~~~~~ &
    \begin{minipage}{.4\textwidth}
      \begin{tabular}{ccl}
	\hline
	Trail & dl & Reason \\
	\hline
	$a_{10}$   & 0 & $( a_{10} )$ \\
	$\neg a_3$ & 1 & Decision \\
	$a_7$      & 2 & Decision \\
	$\neg a_2$ & 3 & Decision \\
	$a_1$      & 4 & Decision \\
	$a_4$      & 4 & $(\neg a_{10} \vee \neg a_1 \vee a_4)$ \\
	$a_5$      & 4 & $(a_3 \vee \neg a_1 \vee a_5)$ \\
	$a_6$      & 4 & $(\neg a_4 \vee a_6)$ \\
	\hline
      \end{tabular}
      \medskip \\
      $\{ \dec{a_{10}}{0}, \dec{\neg a_3}{1}, \dec{a_7}{2}, \dec{\neg a_2}{3}, \dec{a_1}{4}, \dec{a_4}{4}, \dec{a_5}{4}, \dec{a_6}{4}\}$
    \end{minipage}
  \end{tabular}
  \vfill
  \onslide<9>{We say that $(\neg a_{10} \vee a_3 \vee \neg a_1)$ is the {\bf conflict clause} and it is the one we learn}

\end{frame}

\subsection{Non-Chronological Bactracking}

\begin{frame}
  \frametitle{Non-chronological backtracking}

  So far we have been backtracking {\bf chronologically}, but
  \vfill
  given our (wrong) assignment $\{ \dec{a_{10}}{0}, \dec{\neg a_3}{1}, \dec{a_7}{2}, \dec{\neg a_2}{3}, \dec{a_1}{4}, \dec{a_4}{4}, \dec{a_5}{4}, \dec{a_6}{4}\}$
  and the computed conflict clause $(\neg a_{10} \vee a_3 \vee \neg a_1)$, is it really clever to backtrack to level 3 ?
  \vfill
  \pause
  The correct level to backtrack to is: 
  \begin{center}
  ``the level that would have propagated $\neg a_1$ if we had the clause
  $(\neg a_{10} \vee a_3 \vee \neg a_1)$ as part of the original formula''
  \end{center}
  \vfill
  \pause
  In our case, it is level 1, because assignment $\{ \dec{a_{10}}{0}, \dec{\neg a_3}{1} \}$ is
  sufficient to propagate $\neg a_1$ in $(\colone{\neg a_{10}} \vee \colone{a_3} \vee \neg a_1)$
  \vfill
  \pause
  We say that $\neg a_1$ is the {\bf asserting literal} as it becomes true by BCP once 
  we have backtracked to the correct decision level 
\end{frame}

\begin{frame}
  \frametitle{CDCL: Conflict Driven Clause Learning}

  Search
  \vfill

  \begin{tabular}{ccc}
    DPLL                        & & CDCL  \\ 
    no learning                 & & conflict-driven learning \\
    chronological backtracking  & & non-chronological backtracking \\ \\
    \begin{minipage}{.4\textwidth}
      \scalebox{.3}{\input{search_7.pdf_t}}
    \end{minipage}
    & ~~~~ & 
    \begin{minipage}{.4\textwidth}
      \scalebox{.3}{\input{search_8.pdf_t}}
    \end{minipage}
  \end{tabular}

\end{frame}

\begin{frame}
  \frametitle{The CDCL Procedure}

  \scriptsize

  \vfill

  \begin{tabbing}
  as \= a \= a \= a \= as \= asdfsdfasdfasdfasdfasdfasdfasdfasdfasdf \= \kill
  \> $dl = 0$; $trail = \{\ \}$;                              \> \> \> \> \> // Decision level, assignment\\ \\
  \> while ( $true$ ) \\ \\
  \> \> if ( BCP( ) == conflict )                                \> \> \> \> // Do BCP until possible \\
  \> \> \> if ( $dl == 0$ ) return $unsat$;                         \> \> \> // Unresolvable conflict \\
  \> \> \> $C,dl =$ {\sc AnalyzeConflict}( );                           \> \> \> // Compute conf. clause, and dec. level \\
  \> \> \> {\sc AddClause}( $C$ );                                      \> \> \> // Add $C$ to clause database \\
  \> \> \> {\sc BacktrackTo}( $dl$ );                                   \> \> \> // Backtracking (shrinks $trail$) \\
  \> \> else \\
  \> \> \> if ( ``all variables assigned'' ) return $sat$;          \> \> \> // $trail$ holds satisfying assignment \\
  \> \> \> $l = $ {\sc Decision}( );                                      \> \> \> // Do another decision \\
  \> \> \> $trail = trail \cup \{ l \}$ \\                          
  \> \> \> $dl = dl + 1$;                                           \> \> \> // Increase decision level \\
  \end{tabbing}

  \vfill

\end{frame}

\begin{frame}
  \frametitle{The CDCL Procedure}

  \begin{center}
  \scalebox{.45}{\input{cdcl.pdf_t}}
  \end{center}

\end{frame}

\begin{frame}
  \frametitle{Other things which we did not see}

  {\bf Watched Literals}: technique to efficiently discover which clauses
                          become unsat
  \vfill
  {\bf Decision heuristics}: how do we choose the right variable ?
  And which polarity ? ($\top, \bot$). Landmark strategy is VSIDS heuristic
  \vfill
  {\bf Clause removal}: adding too many clauses negatively impacts 
			performance, need mechanisms to remove learnts
  \vfill
  {\bf Restarts}: start the search from scratch, but retaining learnts
  \vfill
  many many other heuristics discovered every year

\end{frame}

\begin{frame}
  \frametitle{Exercise}

  \scriptsize

  Consider the following clause set and assignment
  $$
  \begin{array}{l}
  (\neg a_1 \vee a_2) \\
  (\neg a_1 \vee a_3 \vee a_9) \\
  (\neg a_2 \vee \neg a_3 \vee a_4) \\
  (\neg a_4 \vee a_5 \vee a_{10}) \\
  (\neg a_4 \vee a_6 \vee a_{11}) \\
  (\neg a_5 \vee \neg a_6) \\
  (a_1 \vee a_7 \vee \neg a_{12}) \\
  (a_1 \vee a_8) \\
  (\neg a_7 \vee \neg a_8 \vee \neg a_{13}) \\
  \ldots \\ \\
  \{ \dec{\neg a_9}{1}, \dec{a_{12}}{2}, \dec{a_{13}}{2}, \dec{\neg a_{10}}{3}, \dec{\neg a_{11}}{3}, \ldots, \dec{a_1}{6}\}
  \end{array}
  $$
  \vfill
  \begin{itemize}
    \item Find the correct conflict clause
    \item Find the correct decision level to backtrack
  \end{itemize}

\end{frame}
}
  \end{center}

\end{frame}

\begin{frame}
  \frametitle{Other things which we did not see}

  {\bf Watched Literals}: technique to efficiently discover which clauses
                          become unsat
  \vfill
  {\bf Decision heuristics}: how do we choose the right variable ?
  And which polarity ? ($\top, \bot$). Landmark strategy is VSIDS heuristic
  \vfill
  {\bf Clause removal}: adding too many clauses negatively impacts 
			performance, need mechanisms to remove learnts
  \vfill
  {\bf Restarts}: start the search from scratch, but retaining learnts
  \vfill
  many many other heuristics discovered every year

\end{frame}

\begin{frame}
  \frametitle{Exercise}

  \scriptsize

  Consider the following clause set and assignment
  $$
  \begin{array}{l}
  (\neg a_1 \vee a_2) \\
  (\neg a_1 \vee a_3 \vee a_9) \\
  (\neg a_2 \vee \neg a_3 \vee a_4) \\
  (\neg a_4 \vee a_5 \vee a_{10}) \\
  (\neg a_4 \vee a_6 \vee a_{11}) \\
  (\neg a_5 \vee \neg a_6) \\
  (a_1 \vee a_7 \vee \neg a_{12}) \\
  (a_1 \vee a_8) \\
  (\neg a_7 \vee \neg a_8 \vee \neg a_{13}) \\
  \ldots \\ \\
  \{ \dec{\neg a_9}{1}, \dec{a_{12}}{2}, \dec{a_{13}}{2}, \dec{\neg a_{10}}{3}, \dec{\neg a_{11}}{3}, \ldots, \dec{a_1}{6}\}
  \end{array}
  $$
  \vfill
  \begin{itemize}
    \item Find the correct conflict clause
    \item Find the correct decision level to backtrack
  \end{itemize}

\end{frame}
}
  \end{center}

\end{frame}

\begin{frame}
  \frametitle{Other things which we did not see}

  {\bf Watched Literals}: technique to efficiently discover which clauses
                          become unsat
  \vfill
  {\bf Decision heuristics}: how do we choose the right variable ?
  And which polarity ? ($\top, \bot$). Landmark strategy is VSIDS heuristic
  \vfill
  {\bf Clause removal}: adding too many clauses negatively impacts 
			performance, need mechanisms to remove learnts
  \vfill
  {\bf Restarts}: start the search from scratch, but retaining learnts
  \vfill
  many many other heuristics discovered every year

\end{frame}

\begin{frame}
  \frametitle{Exercise}

  \scriptsize

  Consider the following clause set and assignment
  $$
  \begin{array}{l}
  (\neg a_1 \vee a_2) \\
  (\neg a_1 \vee a_3 \vee a_9) \\
  (\neg a_2 \vee \neg a_3 \vee a_4) \\
  (\neg a_4 \vee a_5 \vee a_{10}) \\
  (\neg a_4 \vee a_6 \vee a_{11}) \\
  (\neg a_5 \vee \neg a_6) \\
  (a_1 \vee a_7 \vee \neg a_{12}) \\
  (a_1 \vee a_8) \\
  (\neg a_7 \vee \neg a_8 \vee \neg a_{13}) \\
  \ldots \\ \\
  \{ \dec{\neg a_9}{1}, \dec{a_{12}}{2}, \dec{a_{13}}{2}, \dec{\neg a_{10}}{3}, \dec{\neg a_{11}}{3}, \ldots, \dec{a_1}{6}\}
  \end{array}
  $$
  \vfill
  \begin{itemize}
    \item Find the correct conflict clause
    \item Find the correct decision level to backtrack
  \end{itemize}

\end{frame}
}
  \end{center}

\end{frame}

\begin{frame}
  \frametitle{Other things which we did not see}

  {\bf Watched Literals}: technique to efficiently discover which clauses
                          become unsat
  \vfill
  {\bf Decision heuristics}: how do we choose the right variable ?
  And which polarity ? ($\top, \bot$). Landmark strategy is VSIDS heuristic
  \vfill
  {\bf Clause removal}: adding too many clauses negatively impacts 
			performance, need mechanisms to remove learnts
  \vfill
  {\bf Restarts}: start the search from scratch, but retaining learnts
  \vfill
  many many other heuristics discovered every year

\end{frame}

\begin{frame}
  \frametitle{Exercise}

  \scriptsize

  Consider the following clause set and assignment
  $$
  \begin{array}{l}
  (\neg a_1 \vee a_2) \\
  (\neg a_1 \vee a_3 \vee a_9) \\
  (\neg a_2 \vee \neg a_3 \vee a_4) \\
  (\neg a_4 \vee a_5 \vee a_{10}) \\
  (\neg a_4 \vee a_6 \vee a_{11}) \\
  (\neg a_5 \vee \neg a_6) \\
  (a_1 \vee a_7 \vee \neg a_{12}) \\
  (a_1 \vee a_8) \\
  (\neg a_7 \vee \neg a_8 \vee \neg a_{13}) \\
  \ldots \\ \\
  \{ \dec{\neg a_9}{1}, \dec{a_{12}}{2}, \dec{a_{13}}{2}, \dec{\neg a_{10}}{3}, \dec{\neg a_{11}}{3}, \ldots, \dec{a_1}{6}\}
  \end{array}
  $$
  \vfill
  \begin{itemize}
    \item Find the correct conflict clause
    \item Find the correct decision level to backtrack
  \end{itemize}

\end{frame}
