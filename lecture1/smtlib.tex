\subsection{SMT-LIB}

\begin{frame}[fragile]
  \frametitle{SMT-LIB (v2) \url{http://www.smtlib.org}}

  The SMT-LIB initiative 
  \begin{itemize}
    \item defines a standard input language for \smtsolvers
    \item defines theories and logics in which \formulae can be written
    \item collects benchmarks 
  \end{itemize}

  \vfill
  \pause
  The SMT-LIB language allows to write \formulae in a lisp-like format. E.g.:
  \begin{verbatim}
    (< (+ x y) 0)
    (= (f x y) (g z))
  \end{verbatim} 
  stand for $x + y < 0$ and $f(x,y) = g(z)$ respectively

  \vfill
  \pause
  An SMT-LIB file looks more similar to a {\bf set of commands} for
  an \smtsolver, rather then a logic formula

\end{frame}

\begin{frame}[fragile]
  \frametitle{SMT-LIB Theories}

  \scriptsize

  An SMT-LIB theory consists of some {\bf sorts}, (e.g., \SInt) 
  and of some functions (e.g., $-,+$). 
  Predicates are also considered functions, with codomain in \SBoo
  (e.g., $<,\leq$). \pause For instance

  \vfill
  \begin{tabular}{cc}
  \begin{minipage}{.4\textwidth}
  \tiny
  \begin{verbatim}
     (theory Ints

      :sorts ((Int 0))
     
      :funs ((NUMERAL Int)
             (- Int Int)
             (- Int Int Int :left-assoc)
             (+ Int Int Int :left-assoc) 
             (* Int Int Int :left-assoc)
             (div Int Int Int :left-assoc)
             (mod Int Int Int)
             (abs Int Int)
             (<= Int Int Bool :chainable)
             (<  Int Int Bool :chainable)
             (>= Int Int Bool :chainable)
             (>  Int Int Bool :chainable)
            )

       [...]
     )
   \end{verbatim}
   \end{minipage}
   &
   \begin{minipage}{.4\textwidth}
     \tiny
      \begin{verbatim}
      (theory Core

       :sorts ((Bool 0))

       :funs ((true Bool)  
              (false Bool)
              (not Bool Bool)
              (=> Bool Bool Bool :right-assoc)
              (and Bool Bool Bool :left-assoc)
              (or Bool Bool Bool :left-assoc)
              (xor Bool Bool Bool :left-assoc)
              (par (A) (= A A Bool :chainable))
              (par (A) (distinct A A Bool :pairwise))
              (par (A) (ite Bool A A A))
             )

       [...]
      )
      \end{verbatim}
      \end{minipage} \\
    \multicolumn{2}{c}{\tiny These definitions can be found at \url{www.smtlib.org}}
  \end{tabular}
  \vfill
  \pause

  The sorts and the function symbols declared in a theory are always {\bf interpreted}.
  This means that a to specify a model for a formula $\varphi$, we just need to specify 
  the assignment of the variables to the concrete values in the sorts.

\end{frame}

\begin{frame}[fragile]
  \frametitle{SMT-LIB Logics}

  \scriptsize
  The difference between ``logic'' and ``theory'' might look very subtle. 
  An SMT-LIB logic includes a theory definition, plus it describes some
  restrictions on how \formulae can be built. \pause

  \vfill
  \hspace{-25pt}
  \begin{tabular}{ccc}
  \begin{minipage}{.4\textwidth}
    \tiny
    \begin{verbatim}
    (logic QF_LIA
     
     :theories (Ints)
     
     :language 
     "Closed quantifier-free formulas built 
     over an arbitrary expansion of the
     Ints signature with free constant symbols, 
     but whose terms of sort Int are all linear, 
     that is, have no occurrences of the function 
     symbols *, /, div, mod, and abs, except as 
     specified the :extensions attribute.
     "
      
     :extensions
     "Terms with _concrete_ coefficients are also 
     allowed, that is, terms of the form c, (* c x), 
     or (* x c)  where x is a free constant and c 
     is a term of the form n or (- n) for some numeral n.
     "
    )
    \end{verbatim}
  \end{minipage}
  & ~~~~ & \pause
  \begin{minipage}{.4\textwidth}
    \tiny
    \begin{verbatim}
    (logic QF_IDL

     :theories (Ints)

     :language
     "Closed quantifier-free formulas with 
     atoms of the form:
     - q
     - (op (- x y) n),
     - (op (- x y) (- n)), or
     - (op x y)
     where
     - q is a variable or free constant symbol of sort Bool,
     - op is <, <=, >, >=, =, or distinct,
     - x, y are free constant symbols of sort Int, 
     - n is a numeral. 
     "
    )
    \end{verbatim}
  \end{minipage}
  \end{tabular}

  \vfill
  \pause

  In the following we will not be so strict, and we will not make any distinction 
  between ``theories'' and ``logics'', calling both ``theories''. \pause 
  For instance when we will say that we reason modulo the theory \Lia we mean 
  that we are working with {\tt QF\_LIA} \formulae

\end{frame}

\begin{frame}[fragile]
  \frametitle{Writing an SMT-LIB file} 
  \scriptsize

  The logic can be specified with the command
  \begin{verbatim}
    (set-logic QF_LIA)
  \end{verbatim}
  \vfill
  \pause
  Variables are declared with
  \begin{verbatim}
    (declare-fun x ( ) Int)
  \end{verbatim}
  \vfill
  \pause
  A formula is specified with 
  \begin{verbatim}
    (assert (<= (+ x y) 0))
  \end{verbatim}
  \vfill
  \pause
  Asks the tool to compute satisfiability of assertions
  \begin{verbatim}
    (check-sat)
  \end{verbatim}
  \vfill
  \pause
  Asks the tool to return a model (in case of sat result)
  \begin{verbatim}
    (set-option :produce-models true)
    ...
    (get-value (x y))
  \end{verbatim}
  \vfill
  \pause
  Disable annoying printouts
  \begin{verbatim}
    (set-option :print-success false)
  \end{verbatim}

\end{frame}

\begin{frame}[fragile]
  \frametitle{Example}

  \begin{verbatim}
    (set-logic QF_LIA)
    (declare-fun x ( ) Int)
    (declare-fun y ( ) Int)
    (declare-fun a ( ) Bool)
    (assert (<= (+ x y) 0))
    (assert (= x 0))
    (assert (or (not a) (= x 1) (>= y 0)))
    (assert (not (= (+ y 1) 0)))
    (check-sat)
    (exit)
  \end{verbatim} 

  which stands for the \Lia formula
  $$
  (x + y \leq 0) \wedge (x = 0) \wedge ((\neg a \vee (x = 1) \vee (y \geq 0)) \wedge \neg(y + 1 = 0)
  $$

\end{frame}
