\subsection{Integers}

\begin{frame}
  \frametitle{Solving \Lia with \Lra}

  The Simplex can be used also to reason (in a complete way) about the integers (\Lia),
  e.g., using known techniques in linear programming
  \vfill
  Given a set of \Lia constraints $S$
  \begin{itemize}
    \item If $S$ is unsatisfiable 
          on \Lra\footnote{This means that we allow variables to assume values in \Rat instead of \Int.}
	  then it is also unsatisfiable on \Lia
    \item If $S$ is satisfiable on \Lra, then we have to check if there is an integer solution
  \end{itemize}
  \vfill\pause
  For the latter case the convex polytope on \Rat is explored sistematically. 
  However, in general, search is necessary: \Lia is NP-Complete, like SAT
  \begin{itemize}
    \item in SAT we split $a$ and $\neg a$, in \Lia we split $x \leq c$ and $x \geq c+1$
    \item in SAT we learn clauses, in \Lia we learn new tableau rows (new constraints)
  \end{itemize}
  \vfill\pause
  When on the integers, $\delta$ is set to $1$ ($\Rat_\delta$ numbers are not used)

\end{frame}
