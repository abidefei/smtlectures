\subsection{Introduction}

\begin{frame}
  \frametitle{Integer Difference Logics (\Idl)}

  The \tatoms of \Idl consists of arithmetic
  constraints of the form
  $$
  x - y \leq c
  $$
  where $x, y$ are variables and $c$ is a numerical constant.
  The domain of $x,y,c$ is that of the integers
  \vfill
  \pause
  Notice that the following translations hold
  \begin{itemize}
    \item $x - y \geq c  \quad\quad\Longrightarrow\quad\quad  y - x \leq -c$  
    \item $x - y < c     \quad\quad\Longrightarrow\quad\quad  x - y \leq  c-1$  
    \item $x - y > c     \quad\quad\Longrightarrow\quad\quad  y - x \leq -c-1$  
    \item $x - y = c     \quad\quad\Longrightarrow\quad\quad (x - y \leq c) \wedge (x - y \geq c)$  
    \item $x - y \not= c \quad\quad\Longrightarrow\quad\quad (x - y < c)    \vee   (x - y > c)$  
  \end{itemize}

\end{frame}

\begin{frame}
  \frametitle{Integer Difference Logic (\Idl)}
  
  \Rdl is similar to \Idl, but it is defined on the rationals. However
  an \Rdl formula can be reduced to an equisatisfiable \Idl formula 
  \vfill
  \pause 
  \Idl/\Rdl can be used to encode a large variety of verification probems
  \begin{itemize}
    \item scheduling
    \item TSP
    \item ASP
    \item timed-automata
    \item sorting algorithms
  \end{itemize}
  \vfill
  Also, the worst-case complexity of solving a conjunction of \Idl constraints is $O(m + n \log n)$

\end{frame}
