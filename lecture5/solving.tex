\subsection{Solving}

\begin{frame}
  \frametitle{Solving}

  Suppose that no negative cycle exists in $G(V,E)$ for $\babst{\mu}$, 
  how do we find a model $\mu$ for the integer variables ?
  \vfill
  \pause
  State-of-the-art methods employ SSSP (Single Source Shortest Paths) algorithms,
  such as the {\bf Bellman-Ford} algorithm (or its variations), as follows:
  \begin{itemize}
    \item we add an artificial vertex $I$ to $V$, and add $\{ (I,x_1;0), \ldots, (I,x_n;0) \}$ to $E$
          (notice that this never creates negative cycles)
    \vfill

    \item Compute the shortest paths from source $I$ (run Bellman-Ford). Let $\pi(x)$ be the shortest path
          from $I$ to $x$, for all $x \in V$
    \vfill

    \item The model $\mu$ can be computed as $\mu(x) = -\pi(x)$ for all $x \in V$ (see later why)
  \end{itemize}

\end{frame}

\begin{frame}
  \frametitle{Bellman-Ford} 

  $\pi(x)$: current distance of $x$ from $I$\\
  $TBV$: queue of vertexes To Be Visited\\

  \begin{tabbing}
  as \= a \= a \= a \= as \= asdfasdfasdfasdfasdfasdfasdfasdf \= \kill
  1  \> $\pi(x) = \infty$ for all $x \in V, x \not= I$ \\
  2  \> $\pi(I) = 0$ \\
  3  \> $TBV.pushBack( I )$ \\
  4  \> while ( $TBV.size(\ ) > 0$ ) \\
  5  \> \> $s = TBV.popFront(\ )$ \\
  6  \> \> foreach $(s,t;w) \in E$         \> \> \> \> // for each outgoing edge \\
  7  \> \> \> if ( $\pi(t) - \pi(s) > w$ )    \> \> \> // is too far ? \\
  8  \> \> \> \> $\pi(t) = \pi(s) + w$           \> \> // relax (decrease $\pi(t)$) \\
  9  \> \> \> \> if ( $TBV.has( t ) == false$ ) \\ 
  10 \> \> \> \> \> $TBV.pushBack( t )$             \> // enqueue t if not there \\
  \end{tabbing}

\end{frame}

\begin{frame}
  \frametitle{Bellman-Ford, Example} 

  \scriptsize
    
  $\pi(x)$: current distance of $x$ from $I$\\
  $TBV$: queue of vertexes To Be Visited\\

  \begin{columns}

  \begin{column}{.45\textwidth}
    \begin{tabbing}
    as \= a \= a \= a \= as \= asdfasdfasdfasdfasdfasdfasdfasdf \= \kill
    \> \coloneat{$\pi(x) = \infty$ for all $x \in V, x \not= I$}{3|handout:0} \\
    \> \coloneat{$\pi(I) = 0$}{3|handout:0} \\
    \> \coloneat{$TBV.pushBack( I )$}{3|handout:0} \\
    \> while ( $TBV.size(\ ) > 0$ ) \\
    \> \> \coloneat{$s = TBV.popFront(\ )$}{4,7,10|handout:0} \\
    \> \> \coloneat{foreach $(s,t;w) \in E$}{5,8,11|handout:0} \\
    \> \> \> \coloneat{if ( $\pi(t) - \pi(s) > w$ )}{5,8,11|handout:0} \\
    \> \> \> \> \coloneat{$\pi(t) = \pi(s) + w$}{5,8,11|handout:0} \\
    \> \> \> \> \coloneat{if ( $TBV.has( t ) == false$ )}{5,8,11|handout:0} \\ 
    \> \> \> \> \> \coloneat{$TBV.pushBack( t )$}{5,8,11|handout:0} \\
    \end{tabbing}

    $TBV = $ [ \only<3|handout:0>{$I$}\only<5,6|handout:0>{$x$}\only<6|handout:0>{$,\,$}\only<6-9|handout:0>{$y,$}\only<6-12|handout:0>{$z,w,t$} ] \\
    Current vertex: \only<4,5|handout:0>{$I$}\only<7,8|handout:0>{$x$}\only<10,11|handout:0>{$y$} \\
    Current edge: \only<5|handout:0>{$(I,x;0)$}\only<11|handout:0>{$(y,z;-1)$} 
  \end{column}

  \begin{column}{.55\textwidth}
    \begin{center}
    \begin{overlayarea}{.55\textwidth}{3cm}
      \only<1|handout:0>{\begin{tikzpicture}[auto]

%
% Styles
%
\tikzstyle{nonvertex} = [circle,draw=white,fill=white,thick,color=white]
\tikzstyle{vertex} = [circle,draw=oproverblue!60,fill=oproverblue!20,thick]
\tikzstyle{edge}   = [->,>=stealth,semithick]
\tikzstyle{weight} = [color=oproveryellow]

\node[vertex] (x) at (  1, 0)        {$x$};
\node[vertex] (y) at (3.5, 0)        {$y$};
\node[vertex] (z) at ( 4,-1.25)      {$z$};
\node[vertex] (w) at (3.5,-2.5)      {$w$};
\node[vertex] (t) at (  1,-2.5)      {$t$};
\node[nonvertex] (I) at (  0, -1.25) {$I$};

\draw[edge] (x) -- node[weight] {$8$}   (y);
\draw[edge] (y) -- node[weight] {$-1$}  (z);
\draw[edge] (z) -- node[weight] {$-6$}  (x);
\draw[edge] (z) -- node[weight] {$2$}   (w);
\draw[edge] (w) -- node[weight] {$-9$}  (x);
\draw[edge] (w) -- node[weight] {$0$}   (t);
\draw[edge] (t) -- node[weight] {$3$}   (x);

\end{tikzpicture}
}
      \only<2|handout:0>{\begin{tikzpicture}[auto]

%
% Styles
%
\tikzstyle{nonvertex} = [circle,draw=white,fill=white,thick,color=white]
\tikzstyle{vertex} = [circle,draw=oproverblue!60,fill=oproverblue!20,thick]
\tikzstyle{edge}   = [->,>=stealth,semithick]
\tikzstyle{edge0}  = [->,>=stealth,semithick,draw=grey]
\tikzstyle{weight} = [color=oproveryellow]

\node[vertex] (x) at (  1, 0)     {$x$};
\node[vertex] (y) at (3.5, 0)     {$y$};
\node[vertex] (z) at ( 4,-1.25)   {$z$};
\node[vertex] (w) at (3.5,-2.5)   {$w$};
\node[vertex] (t) at (  1,-2.5)   {$t$};
\node[vertex] (I) at (  0, -1.25) {$I$};

\draw[edge] (x) -- node[weight] {$8$}   (y);
\draw[edge] (y) -- node[weight] {$-1$}  (z);
\draw[edge] (z) -- node[weight] {$-6$}  (x);
\draw[edge] (z) -- node[weight] {$2$}   (w);
\draw[edge] (w) -- node[weight] {$-9$} (x);
\draw[edge] (w) -- node[weight] {$0$}   (t);
\draw[edge] (t) -- node[weight] {$3$}   (x);

\draw[edge0] (I) -- (x);
\draw[edge0] (I) -- (y);
\draw[edge0] (I) -- (z);
\draw[edge0] (I) -- (w);
\draw[edge0] (I) -- (t);

\end{tikzpicture}
}
      \only<3|handout:0>{\begin{tikzpicture}[auto]

%
% Styles
%
\tikzstyle{nonvertex} = [circle,draw=white,fill=white,thick,color=white]
\tikzstyle{vertex} = [circle,draw=oproverblue!60,fill=oproverblue!20,thick]
\tikzstyle{edge}   = [->,>=stealth,semithick]
\tikzstyle{edge0}  = [->,>=stealth,semithick,draw=grey]
\tikzstyle{weight} = [color=oproveryellow]

\node[vertex] (x) at (  1, 0)     {$\idlnode{x}{\infty}$};
\node[vertex] (y) at (3.5, 0)     {$\idlnode{y}{\infty}$};
\node[vertex] (z) at ( 4,-1.25)   {$\idlnode{z}{\infty}$};
\node[vertex] (w) at (3.5,-2.5)   {$\idlnode{w}{\infty}$};
\node[vertex] (t) at (  1,-2.5)   {$\idlnode{t}{\infty}$};
\node[vertex] (I) at (  0, -1.25) {$\idlnode{I}{0}$};

\draw[edge0] (I) -- (x);
\draw[edge0] (I) -- (y);
\draw[edge0] (I) -- (z);
\draw[edge0] (I) -- (w);
\draw[edge0] (I) -- (t);

\draw[edge] (x) -- node[weight] {$8$}   (y);
\draw[edge] (y) -- node[weight] {$-1$}  (z);
\draw[edge] (z) -- node[weight] {$-6$}  (x);
\draw[edge] (z) -- node[weight] {$2$}   (w);
\draw[edge] (w) -- node[weight] {$-9$}  (x);
\draw[edge] (w) -- node[weight] {$0$}   (t);
\draw[edge] (t) -- node[weight] {$3$}   (x);

\end{tikzpicture}
}
      \only<4|handout:0>{\begin{tikzpicture}[auto]

%
% Styles
%
\tikzstyle{nonvertex} = [circle,draw=white,fill=white,thick,color=white]
\tikzstyle{vertex} = [circle,draw=oproverblue!80,fill=oproverblue!20,thick]
\tikzstyle{edge}   = [->,>=stealth,semithick]
\tikzstyle{edge0}  = [->,>=stealth,semithick,draw=grey]
\tikzstyle{weight} = [color=oproveryellow]
\tikzstyle{hlvertex} = [circle,draw=mydarkgreen!60,fill=mydarkgreen!20,thick]

\node[vertex] (x) at (  1, 0)     {$\idlnode{x}{\infty}$};
\node[vertex] (y) at (3.5, 0)     {$\idlnode{y}{\infty}$};
\node[vertex] (z) at ( 4,-1.25)   {$\idlnode{z}{\infty}$};
\node[vertex] (w) at (3.5,-2.5)   {$\idlnode{w}{\infty}$};
\node[vertex] (t) at (  1,-2.5)   {$\idlnode{t}{\infty}$};
\node[hlvertex] (I) at (  0, -1.25) {$\idlnode{I}{0}$};

\draw[edge0] (I) -- (x);
\draw[edge0] (I) -- (y);
\draw[edge0] (I) -- (z);
\draw[edge0] (I) -- (w);
\draw[edge0] (I) -- (t);

\draw[edge] (x) -- node[weight] {$8$}   (y);
\draw[edge] (y) -- node[weight] {$-1$}  (z);
\draw[edge] (z) -- node[weight] {$-6$}  (x);
\draw[edge] (z) -- node[weight] {$2$}   (w);
\draw[edge] (w) -- node[weight] {$-9$}  (x);
\draw[edge] (w) -- node[weight] {$0$}   (t);
\draw[edge] (t) -- node[weight] {$3$}   (x);

\end{tikzpicture}
}
      \only<5|handout:0>{\begin{tikzpicture}[auto]

%
% Styles
%
\tikzstyle{nonvertex} = [circle,draw=white,fill=white,thick,color=white]
\tikzstyle{vertex} = [circle,draw=oproverblue!80,fill=oproverblue!20,thick]
\tikzstyle{edge}   = [->,>=stealth,semithick]
\tikzstyle{edge0}  = [->,>=stealth,semithick,draw=grey]
\tikzstyle{weight} = [color=oproveryellow]
\tikzstyle{hlvertex} = [circle,draw=mydarkgreen!60,fill=mydarkgreen!20,thick]

\node[vertex] (x) at (  1, 0)     {$\idlnode{x}{0}$};
\node[vertex] (y) at (3.5, 0)     {$\idlnode{y}{\infty}$};
\node[vertex] (z) at ( 4,-1.25)   {$\idlnode{z}{\infty}$};
\node[vertex] (w) at (3.5,-2.5)   {$\idlnode{w}{\infty}$};
\node[vertex] (t) at (  1,-2.5)   {$\idlnode{t}{\infty}$};
\node[hlvertex] (I) at (  0, -1.25) {$\idlnode{I}{0}$};

\draw[edge0] (I) -- (x);
\draw[edge0] (I) -- (y);
\draw[edge0] (I) -- (z);
\draw[edge0] (I) -- (w);
\draw[edge0] (I) -- (t);

\draw[edge] (x) -- node[weight] {$8$}   (y);
\draw[edge] (y) -- node[weight] {$-1$}  (z);
\draw[edge] (z) -- node[weight] {$-6$}  (x);
\draw[edge] (z) -- node[weight] {$2$}   (w);
\draw[edge] (w) -- node[weight] {$-9$}  (x);
\draw[edge] (w) -- node[weight] {$0$}   (t);
\draw[edge] (t) -- node[weight] {$3$}   (x);

\end{tikzpicture}
}
      \only<6|handout:0>{\begin{tikzpicture}[auto]

%
% Styles
%
\tikzstyle{nonvertex} = [circle,draw=white,fill=white,thick,color=white]
\tikzstyle{vertex} = [circle,draw=oproverblue!80,fill=oproverblue!20,thick]
\tikzstyle{edge}   = [->,>=stealth,semithick]
\tikzstyle{edge0}  = [->,>=stealth,semithick,draw=grey]
\tikzstyle{weight} = [color=oproveryellow]
\tikzstyle{hlvertex} = [circle,draw=mydarkgreen!60,fill=mydarkgreen!20,thick]

\node[vertex] (x) at (  1, 0)     {$\idlnode{x}{0}$};
\node[vertex] (y) at (3.5, 0)     {$\idlnode{y}{0}$};
\node[vertex] (z) at ( 4,-1.25)   {$\idlnode{z}{0}$};
\node[vertex] (w) at (3.5,-2.5)   {$\idlnode{w}{0}$};
\node[vertex] (t) at (  1,-2.5)   {$\idlnode{t}{0}$};
\node[vertex] (I) at (  0, -1.25) {$\idlnode{I}{0}$};

\draw[edge0] (I) -- (x);
\draw[edge0] (I) -- (y);
\draw[edge0] (I) -- (z);
\draw[edge0] (I) -- (w);
\draw[edge0] (I) -- (t);

\draw[edge] (x) -- node[weight] {$8$}   (y);
\draw[edge] (y) -- node[weight] {$-1$}  (z);
\draw[edge] (z) -- node[weight] {$-6$}  (x);
\draw[edge] (z) -- node[weight] {$2$}   (w);
\draw[edge] (w) -- node[weight] {$-9$}  (x);
\draw[edge] (w) -- node[weight] {$0$}   (t);
\draw[edge] (t) -- node[weight] {$3$}   (x);

\end{tikzpicture}
}
      \only<7,8|handout:0>{\begin{tikzpicture}[auto]

%
% Styles
%
\tikzstyle{nonvertex} = [circle,draw=white,fill=white,thick,color=white]
\tikzstyle{vertex} = [circle,draw=oproverblue!80,fill=oproverblue!20,thick]
\tikzstyle{edge}   = [->,>=stealth,semithick]
\tikzstyle{edge0}  = [->,>=stealth,semithick,draw=grey]
\tikzstyle{weight} = [color=oproveryellow]
\tikzstyle{hlvertex} = [circle,draw=mydarkgreen!60,fill=mydarkgreen!20,thick]

\node[hlvertex] (x) at (  1, 0)     {$\idlnode{x}{0}$};
\node[vertex] (y) at (3.5, 0)     {$\idlnode{y}{0}$};
\node[vertex] (z) at ( 4,-1.25)   {$\idlnode{z}{0}$};
\node[vertex] (w) at (3.5,-2.5)   {$\idlnode{w}{0}$};
\node[vertex] (t) at (  1,-2.5)   {$\idlnode{t}{0}$};
\node[vertex] (I) at (  0, -1.25) {$\idlnode{I}{0}$};

\draw[edge0] (I) -- (x);
\draw[edge0] (I) -- (y);
\draw[edge0] (I) -- (z);
\draw[edge0] (I) -- (w);
\draw[edge0] (I) -- (t);

\draw[edge] (x) -- node[weight] {$8$}   (y);
\draw[edge] (y) -- node[weight] {$-1$}  (z);
\draw[edge] (z) -- node[weight] {$-6$}  (x);
\draw[edge] (z) -- node[weight] {$2$}   (w);
\draw[edge] (w) -- node[weight] {$-9$}  (x);
\draw[edge] (w) -- node[weight] {$0$}   (t);
\draw[edge] (t) -- node[weight] {$3$}   (x);

\end{tikzpicture}
}
      \only<9|handout:0>{\begin{tikzpicture}[auto]

%
% Styles
%
\tikzstyle{nonvertex} = [circle,draw=white,fill=white,thick,color=white]
\tikzstyle{vertex} = [circle,draw=oproverblue!80,fill=oproverblue!20,thick]
\tikzstyle{edge}   = [->,>=stealth,semithick]
\tikzstyle{edge0}  = [->,>=stealth,semithick,draw=grey]
\tikzstyle{weight} = [color=oproveryellow]
\tikzstyle{hlvertex} = [circle,draw=mydarkgreen!60,fill=mydarkgreen!20,thick]

\node[vertex] (x) at (  1, 0)     {$\idlnode{x}{0}$};
\node[vertex] (y) at (3.5, 0)     {$\idlnode{y}{0}$};
\node[vertex] (z) at ( 4,-1.25)   {$\idlnode{z}{0}$};
\node[vertex] (w) at (3.5,-2.5)   {$\idlnode{w}{0}$};
\node[vertex] (t) at (  1,-2.5)   {$\idlnode{t}{0}$};
\node[vertex] (I) at (  0, -1.25) {$\idlnode{I}{0}$};

\draw[edge0] (I) -- (x);
\draw[edge0] (I) -- (y);
\draw[edge0] (I) -- (z);
\draw[edge0] (I) -- (w);
\draw[edge0] (I) -- (t);

\draw[edge] (x) -- node[weight] {$8$}   (y);
\draw[edge] (y) -- node[weight] {$-1$}  (z);
\draw[edge] (z) -- node[weight] {$-6$}  (x);
\draw[edge] (z) -- node[weight] {$2$}   (w);
\draw[edge] (w) -- node[weight] {$-9$}  (x);
\draw[edge] (w) -- node[weight] {$0$}   (t);
\draw[edge] (t) -- node[weight] {$3$}   (x);

\end{tikzpicture}
}
      \only<10|handout:0>{\begin{tikzpicture}[auto]

%
% Styles
%
\tikzstyle{nonvertex} = [circle,draw=white,fill=white,thick,color=white]
\tikzstyle{vertex} = [circle,draw=oproverblue!80,fill=oproverblue!20,thick]
\tikzstyle{edge}   = [->,>=stealth,semithick]
\tikzstyle{edge0}  = [->,>=stealth,semithick,draw=grey]
\tikzstyle{weight} = [color=oproveryellow]
\tikzstyle{hlvertex} = [circle,draw=mydarkgreen!60,fill=mydarkgreen!20,thick]

\node[vertex] (x) at (  1, 0)     {$\idlnode{x}{0}$};
\node[hlvertex] (y) at (3.5, 0)   {$\idlnode{y}{0}$};
\node[vertex] (z) at ( 4,-1.25)   {$\idlnode{z}{0}$};
\node[vertex] (w) at (3.5,-2.5)   {$\idlnode{w}{0}$};
\node[vertex] (t) at (  1,-2.5)   {$\idlnode{t}{0}$};
\node[vertex] (I) at (  0, -1.25) {$\idlnode{I}{0}$};

\draw[edge0] (I) -- (x);
\draw[edge0] (I) -- (y);
\draw[edge0] (I) -- (z);
\draw[edge0] (I) -- (w);
\draw[edge0] (I) -- (t);

\draw[edge] (x) -- node[weight] {$8$}   (y);
\draw[edge] (y) -- node[weight] {$-1$}  (z);
\draw[edge] (z) -- node[weight] {$-6$}  (x);
\draw[edge] (z) -- node[weight] {$2$}   (w);
\draw[edge] (w) -- node[weight] {$-9$}  (x);
\draw[edge] (w) -- node[weight] {$0$}   (t);
\draw[edge] (t) -- node[weight] {$3$}   (x);

\end{tikzpicture}
}
      \only<11|handout:0>{\begin{tikzpicture}[auto]

%
% Styles
%
\tikzstyle{nonvertex} = [circle,draw=white,fill=white,thick,color=white]
\tikzstyle{vertex} = [circle,draw=oproverblue!80,fill=oproverblue!20,thick]
\tikzstyle{edge}   = [->,>=stealth,semithick]
\tikzstyle{edge0}  = [->,>=stealth,semithick,draw=grey]
\tikzstyle{weight} = [color=oproveryellow]
\tikzstyle{hlvertex} = [circle,draw=mydarkgreen!60,fill=mydarkgreen!20,thick]

\node[vertex] (x) at (  1, 0)     {$\idlnode{x}{0}$};
\node[hlvertex] (y) at (3.5, 0)   {$\idlnode{y}{0}$};
\node[vertex] (z) at ( 4,-1.25)   {$\idlnode{z}{-1}$};
\node[vertex] (w) at (3.5,-2.5)   {$\idlnode{w}{0}$};
\node[vertex] (t) at (  1,-2.5)   {$\idlnode{t}{0}$};
\node[vertex] (I) at (  0, -1.25) {$\idlnode{I}{0}$};

\draw[edge0] (I) -- (x);
\draw[edge0] (I) -- (y);
\draw[edge0] (I) -- (z);
\draw[edge0] (I) -- (w);
\draw[edge0] (I) -- (t);

\draw[edge] (x) -- node[weight] {$8$}   (y);
\draw[edge] (y) -- node[weight] {$-1$}  (z);
\draw[edge] (z) -- node[weight] {$-6$}  (x);
\draw[edge] (z) -- node[weight] {$2$}   (w);
\draw[edge] (w) -- node[weight] {$-9$}  (x);
\draw[edge] (w) -- node[weight] {$0$}   (t);
\draw[edge] (t) -- node[weight] {$3$}   (x);

\end{tikzpicture}
}
      \only<12|handout:0>{\begin{tikzpicture}[auto]

%
% Styles
%
\tikzstyle{nonvertex} = [circle,draw=white,fill=white,thick,color=white]
\tikzstyle{vertex} = [circle,draw=oproverblue!80,fill=oproverblue!20,thick]
\tikzstyle{edge}   = [->,>=stealth,semithick]
\tikzstyle{edge0}  = [->,>=stealth,semithick,draw=grey]
\tikzstyle{weight} = [color=oproveryellow]
\tikzstyle{hlvertex} = [circle,draw=mydarkgreen!60,fill=mydarkgreen!20,thick]

\node[vertex] (x) at (  1, 0)     {$\idlnode{x}{0}$};
\node[vertex] (y) at (3.5, 0)   {$\idlnode{y}{0}$};
\node[vertex] (z) at ( 4,-1.25)   {$\idlnode{z}{-1}$};
\node[vertex] (w) at (3.5,-2.5)   {$\idlnode{w}{0}$};
\node[vertex] (t) at (  1,-2.5)   {$\idlnode{t}{0}$};
\node[vertex] (I) at (  0, -1.25) {$\idlnode{I}{0}$};

\draw[edge0] (I) -- (x);
\draw[edge0] (I) -- (y);
\draw[edge0] (I) -- (z);
\draw[edge0] (I) -- (w);
\draw[edge0] (I) -- (t);

\draw[edge] (x) -- node[weight] {$8$}   (y);
\draw[edge] (y) -- node[weight] {$-1$}  (z);
\draw[edge] (z) -- node[weight] {$-6$}  (x);
\draw[edge] (z) -- node[weight] {$2$}   (w);
\draw[edge] (w) -- node[weight] {$-9$}  (x);
\draw[edge] (w) -- node[weight] {$0$}   (t);
\draw[edge] (t) -- node[weight] {$3$}   (x);

\end{tikzpicture}
}
      \only<13|handout:0>{\begin{tikzpicture}[auto]

%
% Styles
%
\tikzstyle{nonvertex} = [circle,draw=white,fill=white,thick,color=white]
\tikzstyle{vertex} = [circle,draw=oproverblue!80,fill=oproverblue!20,thick]
\tikzstyle{edge}   = [->,>=stealth,semithick]
\tikzstyle{edge0}  = [->,>=stealth,semithick,draw=grey]
\tikzstyle{weight} = [color=oproveryellow]
\tikzstyle{hlvertex} = [circle,draw=mydarkgreen!60,fill=mydarkgreen!20,thick]

\node[vertex] (x) at (  1, 0)     {$\idlnode{x}{-9}$};
\node[vertex] (y) at (3.5, 0)     {$\idlnode{y}{-1}$};
\node[vertex] (z) at ( 4,-1.25)   {$\idlnode{z}{-2}$};
\node[vertex] (w) at (3.5,-2.5)   {$\idlnode{w}{0}$};
\node[vertex] (t) at (  1,-2.5)   {$\idlnode{t}{0}$};
\node[vertex] (I) at (  0, -1.25) {$\idlnode{I}{0}$};

\draw[edge0] (I) -- (x);
\draw[edge0] (I) -- (y);
\draw[edge0] (I) -- (z);
\draw[edge0] (I) -- (w);
\draw[edge0] (I) -- (t);

\draw[edge] (x) -- node[weight] {$8$}   (y);
\draw[edge] (y) -- node[weight] {$-1$}  (z);
\draw[edge] (z) -- node[weight] {$-6$}  (x);
\draw[edge] (z) -- node[weight] {$2$}   (w);
\draw[edge] (w) -- node[weight] {$-9$}  (x);
\draw[edge] (w) -- node[weight] {$0$}   (t);
\draw[edge] (t) -- node[weight] {$3$}   (x);

\end{tikzpicture}
}
      \only<14>{\begin{tikzpicture}[auto]

%
% Styles
%
\tikzstyle{nonvertex} = [circle,draw=white,fill=white,thick,color=white]
\tikzstyle{vertex} = [circle,draw=oproverblue!80,fill=oproverblue!20,thick]
\tikzstyle{edge}   = [->,>=stealth,semithick]
\tikzstyle{edge0}  = [->,>=stealth,semithick,draw=grey]
\tikzstyle{edge1}  = [->,>=stealth,very thick,draw=red,color=red]
\tikzstyle{weight} = [color=oproveryellow]
\tikzstyle{hlvertex} = [circle,draw=mydarkgreen!60,fill=mydarkgreen!20,thick]

\node[vertex] (x) at (  1, 0)     {$\idlnode{x}{-9}$};
\node[vertex] (y) at (3.5, 0)     {$\idlnode{y}{-1}$};
\node[vertex] (z) at ( 4,-1.25)   {$\idlnode{z}{-2}$};
\node[vertex] (w) at (3.5,-2.5)   {$\idlnode{w}{0}$};
\node[vertex] (t) at (  1,-2.5)   {$\idlnode{t}{0}$};
\node[vertex] (I) at (  0, -1.25) {$\idlnode{I}{0}$};

\draw[edge0] (I) -- (x);
\draw[edge0] (I) -- (y);
\draw[edge0] (I) -- (z);
\draw[edge1] (I) -- (w);
\draw[edge1] (I) -- (t);

\draw[edge1] (x) -- node[weight] {$8$}   (y);
\draw[edge1] (y) -- node[weight] {$-1$}  (z);
\draw[edge] (z) -- node[weight] {$-6$}  (x);
\draw[edge] (z) -- node[weight] {$2$}   (w);
\draw[edge1] (w) -- node[weight] {$-9$}  (x);
\draw[edge] (w) -- node[weight] {$0$}   (t);
\draw[edge] (t) -- node[weight] {$3$}   (x);

\node[circle] (h) at (0,-3.5)  {};
\node[circle] (k) at (2,-3.5)  {$\stackrel{\mbox{shortest paths}}{\mbox{(spanning tree)}}$};

\draw[edge1] (h) -- (k);

\end{tikzpicture}
}
    \end{overlayarea}
    \end{center}
  \end{column}

  \end{columns}

\end{frame}

\begin{frame}
  \frametitle{Bellman-Ford, consideration}

  \scriptsize

  \begin{exampleblock}{Invariant}
  At the end of BF, $\pi(x)$ holds the shortest distance from $I$ to $x$, for all $x \in V$ \\
  \end{exampleblock}
  \vfill
  \pause
  \begin{lemma}[Shortest Path]
  At the end of BF, $\pi(y) - \pi(x) \leq c$ holds for all $(x,y;c) \in E$
  \end{lemma}
  \vfill
  \pause
  \begin{proof}
  Suppose, for the sake of contradiction, that for an edge $(x,y;c)$, we have $\pi(y) - \pi(x) > c$
    \begin{center}
      \begin{tikzpicture}[auto]

%
% Styles
%
\tikzstyle{vertex} = [circle,draw=oproverblue!60,fill=oproverblue!20,thick]
\tikzstyle{edge}   = [->,>=stealth,thick]
\tikzstyle{edge1}  = [->,>=stealth,thick,draw=red,color=red]
\tikzstyle{weight} = [color=oproveryellow]

\node[vertex] (x) at (0,0)   {$\idlnode{x}{\pi(x)}$};
\node[vertex] (y) at (2,0)   {$\idlnode{y}{\pi(y)}$};
\node[vertex] (I) at (-4,0)  {$\idlnode{I}{0}$};
\node[circle] (h) at (-2,0)  {\ldots};
\node (k) at (4,0)   {\ldots};

\draw[edge] (x) -- node[weight] {$c$} (y);
\draw[edge1] (I) -- (h);
\draw[edge1] (h) -- (x);
\draw[edge1] (h) to [out=-45,in=225] (y);
\draw[edge] (y) -- (k);

\node[circle] (l) at (-5,-1)  {};
\node[circle] (m) at (-3,-1)  {shortest paths};

\draw[edge1] (l) -- (m);

\end{tikzpicture}

    \end{center}
    \vspace{-30pt}
    $\pi(x)$ is the shortest dist. from $I$ to $x$ (by Invariant). But since $\pi(y) > \pi(x) + c$, 
    the shortest path from $I$ to $y$ is $\pi(x) + c$. So $\pi(y)$ is not the shortest dist. Contradiction.
  \end{proof}

\end{frame}

\begin{frame}
  \frametitle{Finding a model $\mu$}

  Because of the previous lemma we have that
  \vfill
  \begin{center}
    $\pi(y) - \pi(x) \leq c\quad\quad$ holds for all $(x,y;c) \in E$
  \end{center}
  \vfill
  So, if we take $\quad\quad\mu(x) = -\pi(x)\quad\quad$ we have that
  \vfill
  \begin{center}
    $\mu(x) - \mu(y) \leq c\quad\quad$ holds for all constraints in $\babst{\mu}$
  \end{center}
  \vfill
  and therefore $\mu$ is a model

\end{frame}
