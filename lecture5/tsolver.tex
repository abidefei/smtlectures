\begin{frame}
  \frametitle{\tsolver Compliance}

  So far we have seen that BF can be used
  to compute a model of a given {\bf consistent}
  set of \Idl constraints
  \vfill
  Recall that to have an efficient \tsolver
  the following features should be supported
  \vfill
  \begin{itemize}
    \item Minimal Conflicts
  \vfill
    \item Incrementality
  \vfill
    \item Backtrackability
  \vfill
    \item Theory Propagation
  \end{itemize}
  \vfill
  Let's see how to improve the current algorithm to
  support them

\end{frame}

\subsection{Minimal Conflicts}

\begin{frame}
  \frametitle{Minimal Conflicts}

  So far we assumed that $G(V,E)$ did not contain
  negative cycles. However it is not difficult to
  tweak the BF to recognize them
  \vfill
  \pause
  Recall the {\bf spanning tree}: the tree that holds
  shortest paths 
  \vfill
  \begin{center}
    \begin{tikzpicture}[auto]

%
% Styles
%
\tikzstyle{nonvertex} = [circle,draw=white,fill=white,thick,color=white]
\tikzstyle{vertex} = [circle,draw=oproverblue!80,fill=oproverblue!20,thick]
\tikzstyle{edge}   = [->,>=stealth,semithick,draw=grey,color=grey]
\tikzstyle{edge0}  = [->,>=stealth,semithick,draw=grey]
\tikzstyle{edge1}  = [->,>=stealth,very thick,draw=red,color=red]
\tikzstyle{weight} = [color=oproveryellow]
\tikzstyle{hlvertex} = [circle,draw=mydarkgreen!60,fill=mydarkgreen!20,thick]

\node[vertex] (x) at (  1, 0)     {};
\node[vertex] (y) at (3.5, 0)     {};
\node[vertex] (z) at ( 4,-1.25)   {};
\node[vertex] (w) at (3.5,-2.5)   {};
\node[vertex] (t) at (  1,-2.5)   {};
\node[vertex] (I) at (  0, -1.25) {};

\draw[edge1] (I) -- (x);
\draw[edge]  (I) -- (y);
\draw[edge]  (I) -- (z);
\draw[edge] (I) -- (w);
\draw[edge1] (I) -- (t);

\draw[edge1] (x) -- (y);
\draw[edge] (y) -- (z);
\draw[edge1] (x) -- (z);
\draw[edge]  (z) -- (w);
\draw[edge] (w) -- (x);
\draw[edge1]  (t) -- (w);
\draw[edge]  (t) -- (x);

\end{tikzpicture}

  \end{center}
  \vfill
  \pause
  It is easy to keep the spanning tree:
  \begin{itemize}
    \item each node $t$ keeps two fields {\bf fatherVertex}
	  and {\bf fatherEdge} that stores its father 
          in the spanning tree
    \item when a {\bf relax} is done on $t$ (line 8 of BF)
	  through an edge $e = (s, t; c)$, $t.fatherVertex$ is
	  set to $s$ and $t.fatherEdge$ is set to $e$
  \end{itemize}

\end{frame}

\begin{frame}
  \frametitle{Minimal Conflicts}

  \scriptsize

  When a negative cycle exists, the spanning tree tends
  to become {\bf cyclic} (trees are acyclic instead). It is
  easy to recognize this situation and report the negative cycle

  \vfill
  \pause

  \begin{columns}

  \begin{column}{.45\textwidth}
    $TBV =$ [ ] \\
    Current vetex: $z$ \\
    Current edge: $(z,w;2)$ 
  \end{column}

  \begin{column}{.45\textwidth}
    \begin{center}
    \begin{overlayarea}{\textwidth}{3cm}
      \begin{tikzpicture}[auto]

%
% Styles
%
\tikzstyle{nonvertex} = [circle,draw=white,fill=white,thick,color=white]
\tikzstyle{vertex} = [circle,draw=oproverblue!60,fill=oproverblue!20,thick]
\tikzstyle{edge}   = [->,>=stealth,semithick]
\tikzstyle{edge0}  = [->,>=stealth,semithick,draw=grey]
\tikzstyle{edge1}  = [->,>=stealth,very thick,draw=red,color=red]
\tikzstyle{weight} = [color=oproveryellow]
\tikzstyle{hlvertex} = [circle,draw=mydarkgreen!60,fill=mydarkgreen!20,thick]

\node[vertex] (x) at (  1, 0)     {$\idlnode{x}{-10}$};
\node[vertex] (y) at (3.5, 0)     {$\idlnode{y}{-2}$};
\node[hlvertex] (z) at ( 4,-1.25)   {$\idlnode{z}{-3}$};
\node[vertex] (w) at (3.5,-2.5)   {$\idlnode{w}{0}$};
\node[vertex] (t) at (  1,-2.5)   {$\idlnode{t}{0}$};
\node[vertex] (I) at (  0, -1.25) {$\idlnode{I}{0}$};

\draw[edge1] (x) -- node[weight] {$8$}   (y);
\draw[edge1] (y) -- node[weight] {$-1$}  (z);
\draw[edge]  (z) -- node[weight] {}  (x);
\draw[edge]  (z) -- node[weight] {$2$}   (w);
\draw[edge1] (w) -- node[weight] {$-10$} (x);
\draw[edge]  (w) -- node[weight] {}   (t);
\draw[edge]  (t) -- node[weight] {}   (x);

\draw[edge0] (I) -- (x);
\draw[edge0] (I) -- (y);
\draw[edge0] (I) -- (z);
\draw[edge1] (I) -- (w);
\draw[edge1] (I) -- (t);

\end{tikzpicture}

    \end{overlayarea}
    \end{center}
  \end{column}

  \end{columns}

  \vfill
  \pause

  \begin{tabbing}
  as \= a \= a \= a \= as \= asdfasdfasdfasdfasdfasdfasdfasdf \= \kill
  checkNegativeCycle( $s$, $t$ ) \\
  \> $\babst{\nu} = \{ \ \}$ \\ \\
  \> while ( $s \not= t$ \&\& $s \not= I$ ) \\
  \> \> $s = s.fatherVertex$ \\
  \> \> $\babst{\nu} = \babst{\nu} \cup \{ s.fatherEdge \}$ \\ \\
  \> if ( $s$ == $t$ ) return conflict \>\>\>\>\> // Neg. Cycle detected \\
  \> return OK        \>\>\>\>\> // $I$ reached
  \end{tabbing}

\end{frame}

\begin{frame}
  \frametitle{Minimal Conflicts}

  Bellman-Ford with negative cycle detection

  \begin{tabbing}
  as \= a \= a \= a \= as \= asdfasdfasdfasdfasdfasdfasdfasdf \= \kill
  1  \> $\pi(x) = \infty$ for all $x \in V, x \not= I$ \\
  2  \> $\pi(I) = 0$ \\
  3  \> $TBV.pushBack( I )$ \\
  4  \> while ( $TBV.size(\ ) > 0$ ) \\
  5  \> \> $s = TBV.popFront(\ )$ \\
  6  \> \> foreach $(s,t;w) \in E$         \> \> \> \> // for each outgoing edge \\
  7  \> \> \> if ( $\pi(t) - \pi(s) > w$ )    \> \> \> // is too far ? \\
  8  \> \> \> \> \colone{if ( checkNegativeCycle( $s$, $t$ ) == conflict )} \\
  9  \> \> \> \> \> \colone{return $\babst{\nu}$} \\
  10 \> \> \> \> \colone{$s.fatherVertex = t$} \\
  11 \> \> \> \> \colone{$s.fatherEdge = (s,t;w)$} \\
  12  \> \> \> \> $\pi(t) = \pi(s) + w$           \> \> // relax (decrease $\pi(t)$) \\
  13  \> \> \> \> if ( $TBV.has( t ) == false$ ) \\ 
  14 \> \> \> \> \> $TBV.pushBack( t )$             \> // enqueue $t$ if not there \\
  \end{tabbing}

\end{frame}

\subsection{Incrementality}

\begin{frame}
  \frametitle{Incrementality}

  \scriptsize
  
  Incrementality comes almost for free !
  \vfill
  \begin{itemize}

    \item we always keep the $\pi$ function ``alive'', and only
          update it slightly
  \vfill

    \item $G(V,E)$ keeps track of {\bf active} and {\bf inactive}
          edges: active edges are those on $\babst{\mu}$
  \vfill

    \item when a new \tatom $x - y \leq c$ is added to $\babst{\mu}$, the
	  corresponding edge $(x,y;c)$ becomes active in the graph
  \vfill

    \item we add $x$ to $TBV$, and run BF from line 4

  \end{itemize}
  \vfill
  \begin{columns}

  \begin{column}{.3\textwidth}
  \begin{center}
    $x-y \leq 8 \not\in \babst{\mu}$ \\
    \medskip
    \scalebox{.7}{\begin{tikzpicture}[auto]

%
% Styles
%
\tikzstyle{nonvertex} = [circle,draw=white,fill=white,thick,color=white]
\tikzstyle{vertex} = [circle,draw=oproverblue!80,fill=oproverblue!20,thick]
\tikzstyle{edge}   = [->,>=stealth,semithick]
\tikzstyle{edge0}  = [->,>=stealth,semithick,draw=grey]
\tikzstyle{edge1}  = [->,>=stealth,very thick,draw=red,color=red]
\tikzstyle{edge2}  = [->,>=stealth,semithick,draw=grey,color=grey,dashed]
\tikzstyle{weight} = [color=oproveryellow]
\tikzstyle{hlvertex} = [circle,draw=mydarkgreen!60,fill=mydarkgreen!20,thick]

\node[vertex] (x) at (  1, 0)     {$\idlnode{x}{-9}$};
\node[vertex] (y) at (3.5, 0)     {$\idlnode{y}{0}$};
\node[vertex] (z) at ( 4,-1.25)   {$\idlnode{z}{-1}$};
\node[vertex] (w) at (3.5,-2.5)   {$\idlnode{w}{0}$};
\node[vertex] (t) at (  1,-2.5)   {$\idlnode{t}{0}$};
\node[vertex] (I) at (  0, -1.25) {$\idlnode{I}{0}$};

\draw[edge0] (I) -- (x);
\draw[edge1] (I) -- (y);
\draw[edge0] (I) -- (z);
\draw[edge1] (I) -- (w);
\draw[edge1] (I) -- (t);

\draw[edge2] (x) -- node[weight] {$8$}   (y);
\draw[edge1] (y) -- node[weight] {$-1$}  (z);
\draw[edge] (z) -- node[weight] {$-6$}  (x);
\draw[edge] (z) -- node[weight] {$2$}   (w);
\draw[edge1] (w) -- node[weight] {$-9$}  (x);
\draw[edge] (w) -- node[weight] {$0$}   (t);
\draw[edge] (t) -- node[weight] {$3$}   (x);

\end{tikzpicture}
}
  \end{center} 
  \end{column}

  \pause

  \begin{column}{.3\textwidth}
  \begin{center}
    $x-y \leq 8$ pushed to $\babst{\mu}$ \\
    \medskip
    \scalebox{.7}{\begin{tikzpicture}[auto]

%
% Styles
%
\tikzstyle{nonvertex} = [circle,draw=white,fill=white,thick,color=white]
\tikzstyle{vertex} = [circle,draw=oproverblue!80,fill=oproverblue!20,thick]
\tikzstyle{edge}   = [->,>=stealth,semithick]
\tikzstyle{edge0}  = [->,>=stealth,semithick,draw=grey]
\tikzstyle{edge1}  = [->,>=stealth,very thick,draw=red,color=red]
\tikzstyle{edge2}  = [->,>=stealth,semithick,draw=grey,color=grey,dashed]
\tikzstyle{weight} = [color=oproveryellow]
\tikzstyle{hlvertex} = [circle,draw=mydarkgreen!60,fill=mydarkgreen!20,thick]

\node[hlvertex] (x) at (  1, 0)     {$\idlnode{x}{-9}$};
\node[vertex] (y) at (3.5, 0)     {$\idlnode{y}{0}$};
\node[vertex] (z) at ( 4,-1.25)   {$\idlnode{z}{-1}$};
\node[vertex] (w) at (3.5,-2.5)   {$\idlnode{w}{0}$};
\node[vertex] (t) at (  1,-2.5)   {$\idlnode{t}{0}$};
\node[vertex] (I) at (  0, -1.25) {$\idlnode{I}{0}$};

\draw[edge0] (I) -- (x);
\draw[edge1] (I) -- (y);
\draw[edge0] (I) -- (z);
\draw[edge1] (I) -- (w);
\draw[edge1] (I) -- (t);

\draw[edge] (x) -- node[weight] {$8$}   (y);
\draw[edge1] (y) -- node[weight] {$-1$}  (z);
\draw[edge] (z) -- node[weight] {$-6$}  (x);
\draw[edge] (z) -- node[weight] {$2$}   (w);
\draw[edge1] (w) -- node[weight] {$-9$}  (x);
\draw[edge] (w) -- node[weight] {$0$}   (t);
\draw[edge] (t) -- node[weight] {$3$}   (x);

\end{tikzpicture}
}
  \end{center} 
  \end{column}

  \pause

  \begin{column}{.3\textwidth}
  \begin{center}
    After BF \\
    \medskip
    \scalebox{.7}{\begin{tikzpicture}[auto]

%
% Styles
%
\tikzstyle{nonvertex} = [circle,draw=white,fill=white,thick,color=white]
\tikzstyle{vertex} = [circle,draw=oproverblue!80,fill=oproverblue!20,thick]
\tikzstyle{edge}   = [->,>=stealth,semithick]
\tikzstyle{edge0}  = [->,>=stealth,semithick,draw=grey]
\tikzstyle{edge1}  = [->,>=stealth,very thick,draw=red,color=red]
\tikzstyle{edge2}  = [->,>=stealth,semithick,draw=grey,color=grey,dashed]
\tikzstyle{weight} = [color=oproveryellow]
\tikzstyle{hlvertex} = [circle,draw=mydarkgreen!60,fill=mydarkgreen!20,thick]
\tikzstyle{hlvertex1} = [circle,draw=oproveryellow,fill=oproveryellow!50,thick]

\node[hlvertex1] (x) at (  1, 0)     {$\idlnode{x}{-9}$};
\node[hlvertex1] (y) at (3.5, 0)     {$\idlnode{y}{-1}$};
\node[hlvertex1] (z) at ( 4,-1.25)   {$\idlnode{z}{-2}$};
\node[vertex] (w) at (3.5,-2.5)   {$\idlnode{w}{0}$};
\node[vertex] (t) at (  1,-2.5)   {$\idlnode{t}{0}$};
\node[vertex] (I) at (  0, -1.25) {$\idlnode{I}{0}$};

\draw[edge0] (I) -- (x);
\draw[edge0] (I) -- (y);
\draw[edge0] (I) -- (z);
\draw[edge1] (I) -- (w);
\draw[edge1] (I) -- (t);

\draw[edge1] (x) -- node[weight] {$8$}   (y);
\draw[edge1] (y) -- node[weight] {$-1$}  (z);
\draw[edge] (z) -- node[weight] {$-6$}  (x);
\draw[edge] (z) -- node[weight] {$2$}   (w);
\draw[edge1] (w) -- node[weight] {$-9$}  (x);
\draw[edge] (w) -- node[weight] {$0$}   (t);
\draw[edge] (t) -- node[weight] {$3$}   (x);

\end{tikzpicture}
}
  \end{center} 
  \end{column}

  \end{columns}

\end{frame}

\subsection{Backtrackability}

\begin{frame}
  \frametitle{Backtrackability}

  Backtracking can be done efficiently
  \vfill
  First of all, recall that $\mu(x) = -\pi(x)$. Second obeserve that
  \vfill
  \begin{exampleblock}{Observation 1}
    Let $\mu$ be a model for a set $\babst{\mu}$ of constraints.
    Then $\mu$ is also a model for {\bf any subset} of $\babst{\mu}$
  \end{exampleblock}
  \vfill
  Observation 1 implies that when backtracking we just have to turn some
  edges into inactive, and keep the last $\pi$ intact
  \vfill
  This is done as follows: BF will always work on a temporary $\pi$, called $\pi'$. If satisfiable,
  $\pi$ is replaced by $\pi'$, otherwise we keep $\pi$  

\end{frame}

\subsection{Theory Propagation}

\begin{frame}
  \frametitle{Theory Propagation}

  Theory Propagation is the process of activating some edges in the
  graph that are implied by the current $\babst{\mu}$
  \vfill
  \begin{exampleblock}{Observation 2}
    A set of constraints \\
    $\{\ \colone{x_1} - x_2 \leq c_1,\ x_2 - x_3 \leq c_2,\ \ldots,\ x_{n-1} - \coltwo{x_n} \leq c_{n-1}\ \}$ \\
    implies 
    $\quad\quad\colone{x_1} - \coltwo{x_n} \leq c_n\quad\quad$ 
    iff 
    $\quad\quad c_1 + c_2 + \ldots c_{n-1} \leq c_n$
  \end{exampleblock}
  \vfill
  \pause
  Example:

  \begin{center}
    \begin{tikzpicture}[auto]

%
% Styles
%
\tikzstyle{vertex} = [circle,draw=oproverblue!60,fill=oproverblue!20,thick]
\tikzstyle{edge}   = [->,>=stealth,thick]
\tikzstyle{edge2}  = [->,>=stealth,semithick,draw=grey,color=grey,dashed]
\tikzstyle{weight} = [color=oproveryellow]

\node[vertex] (x) at (0,0)   {$x$};
\node[vertex] (y) at (2,0)   {$y$};
\node[vertex] (z) at (4,0)   {$z$};
\node[circle] (h) at (-2,0)   {\ldots};
\node[circle] (k) at (6,0)    {\ldots};

\draw[edge]  (x) -- node[weight] {$4$} (y);
\draw[edge]  (y) -- node[weight] {$2$} (z);
\draw[edge]  (h) -- (x);
\draw[edge]  (z) -- (k);
\draw[edge2] (x) to [out=-45,in=225] node[weight] {$10$} (z);

\end{tikzpicture}
 
  \end{center}

  $x - z \leq 10$, currently inactive, can be theory-propagated

\end{frame}
