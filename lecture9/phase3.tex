\begin{frame}[fragile]
  \frametitle{Phase 3 - Implementing the solver}
  \framesubtitle{Playing with the Enodes}

  \scriptsize

  The \verb|Enode| is the data structure that stores any
  term and formula
  \vfill
  There are three kinds of Enode
  \vfill
  \begin{columns}

    \begin{column}{.2\textwidth}
      \begin{center}
      Symbol
      \bigskip\\
      \begin{tikzpicture}[auto]

%
% Styles
%
\tikzstyle{symbol} = [circle,draw=mydarkred!60,fill=mydarkred!20,thick]
\tikzstyle{term}   = [circle,draw=oproverblue!60,fill=oproverblue!20,thick]
\tikzstyle{list}   = [circle,draw=mydarkgreen!60,fill=mydarkgreen!20,thick]

\tikzstyle{edge}   = [->,>=stealth,thick]

\node[symbol] (x) at ( -1, 0)  {$+$};
%\node[term]   (y) at (  0, 1)  {};
%\node[list]   (z) at (  1, 0)  {};

%\draw[edge] (y) -- (x);
%\draw[edge] (y) -- (z);

\end{tikzpicture}
 
      \end{center}
    \end{column}

    \begin{column}{.3\textwidth}
      \begin{center}
      Term
      \bigskip\\
      \begin{tikzpicture}[auto]

%
% Styles
%
\tikzstyle{symbol} = [circle,draw=mydarkred!60,fill=mydarkred!20,thick]
\tikzstyle{term}   = [circle,draw=oproverblue!60,fill=oproverblue!20,thick, text width=10pt]
\tikzstyle{list}   = [circle,draw=mydarkgreen!60,fill=mydarkgreen!20,thick, text width=10pt]

\tikzstyle{edge}   = [->,>=stealth,thick]

\node[symbol] (x) at ( -1, 0)  {$+$};
\node[term]   (y) at (  0, 1)  {};
\node[list]   (z) at (  1, 0)  {};

\draw[edge] (y) -- (x);
\draw[edge] (y) -- (z);

\end{tikzpicture}
 
      \end{center}
    \end{column}

    \begin{column}{.45\textwidth}
      \begin{center}
      List
      \bigskip\\
      \begin{tikzpicture}[auto]

%
% Styles
%
\tikzstyle{symbol} = [circle,draw=mydarkred!60,fill=mydarkred!20,thick]
\tikzstyle{term}   = [circle,draw=oproverblue!60,fill=oproverblue!20,thick, text width=10pt]
\tikzstyle{list}   = [circle,draw=mydarkgreen!60,fill=mydarkgreen!20,thick, text width=10pt, align=center]

\tikzstyle{edge}   = [->,>=stealth,thick]

\node[term]   (x)   at ( -1, 0)   {};
\node[list]   (y)   at (  0, 1)   {};
\node[list]   (nil) at (  2.5, 0.5) {-};
\node[list]   (z)   at (  1, 0)   {};

\draw[edge] (y) -- (x);
\draw[edge] (y) -- (z);

\end{tikzpicture}
 
      \end{center}
    \end{column}

  \end{columns}

  \vfill\pause

  \begin{columns}

    \begin{column}{.5\textwidth}
      E.g. the term $x \leq y$ is represented as
      \bigskip\\
      If \verb|e| is the Enode for $x \leq y$, we retrieve
      the Enode for \verb|x| with\\
      \verb|Enode * lhs = e->get1st( )|
      \medskip\\
      and the Enode for \verb|y| with\\
      \verb|Enode * rhs = e->get2nd( )|
    \end{column}

    \begin{column}{.5\textwidth}

      \begin{center}
        \scalebox{.7}{\begin{tikzpicture}[auto]

%
% Styles
%
\tikzstyle{symbol} = [circle,draw=mydarkred!60,fill=mydarkred!20,thick, text width=10pt, align=center]
\tikzstyle{term}   = [circle,draw=oproverblue!60,fill=oproverblue!20,thick, text width=10pt]
\tikzstyle{list}   = [circle,draw=mydarkgreen!60,fill=mydarkgreen!20,thick, text width=10pt, align=center]

\tikzstyle{edge}   = [->,>=stealth,thick]

\node[symbol]   (s1) at (  0,  0)  {$\leq$};
\node[symbol]   (s2) at (  0, -2)  {$x$};
\node[symbol]   (s3) at (  1, -3)  {$y$};
\node[term]     (t2) at (  1, -1)  {};
\node[term]     (t3) at (  2, -2)  {};
\node[term]     (t)  at (  1,  1)  {};
\node[list]     (l2) at (  2,  0)  {};
\node[list]     (l1) at (  3, -1)  {};
\node[list]     (l0) at (  4, -2)  {-};

\draw[edge]  (t) -- (s1);
\draw[edge]  (t) -- (l2);
\draw[edge] (l2) -- (t2);
\draw[edge] (l2) -- (l1);
\draw[edge] (l1) -- (t3);
\draw[edge] (l1) -- (l0);
\draw[edge] (t2) -- (l0);
\draw[edge] (t3) -- (l0);
\draw[edge] (t2) -- (s2);
\draw[edge] (t3) -- (s3);

\end{tikzpicture}
}
      \end{center}

    \end{column}

  \end{columns}

\end{frame}

\begin{frame}[fragile]
  \frametitle{Phase 3 - Implementing the solver}
  \framesubtitle{Playing with the Enodes}

  The polarity of an Enode can be retrieved with
  \begin{center}
  \verb|lbool sign = e->getPolarity( );|
  \end{center}
  and it could be \verb|l_True| or \verb|l_False|
  \vfill
  An Enode \verb|e| can be simply printed with
  \begin{center}
  \verb|cerr << "printing enode: " << e << endl;|
  \end{center}

\end{frame}

\begin{frame}[fragile]
  \frametitle{Phase 3 - Implementing the solver}

  A set of constraints is unsatisfiable iff there is a cycle 
  $$x \leq y,\ y \leq z,\ \ldots,\ w \leq x$$  
  \vfill
  A model is therefore is any ``acyclic'' set of constraints
  \vfill

  \begin{columns}

    \begin{column}{.3\textwidth}

      \begin{overlayarea}{.3\textwidth}{40pt}

	$x \leq y$ \\
	$y \leq z$ \\
	$z \leq x$

      \end{overlayarea}
      
    \end{column}
    
    \begin{column}{.4\textwidth}

      \begin{overlayarea}{.4\textwidth}{40pt}
      \begin{tikzpicture}[auto]

%
% Styles
%
\tikzstyle{vertex} = [circle,draw=oproverblue!60,fill=oproverblue!20,thick]
\tikzstyle{edge}   = [->,>=stealth,thick]

\node[vertex] (x) at ( 0, 0)  {$x$};
\node[vertex] (y) at ( 2, 0)  {$y$};
\node[vertex] (z) at ( 4, 0)  {$z$};

\draw[edge] (x) -- (y);
\draw[edge] (y) -- (z);
\draw[edge] (z) to [out=225,in=-45] (x);

\end{tikzpicture}

      \end{overlayarea}
      
    \end{column}

    \begin{column}{.2\textwidth}

      unsat

    \end{column}

  \end{columns}

  \vfill
  Therefore we need to

  \begin{enumerate}

    \item Represent the graph from the received constraints

    \item Check if the graph contains a cycle

  \end{enumerate}

\end{frame}

\begin{frame}[fragile]
  \frametitle{Phase 3 - Implementing the solver}
  \framesubtitle{Representing the graph}

  We represent the graph with {\bf adjacency lists}, i.e., every node (variable) is
  assigned to a list of outgoing edges
  \vfill
  We use STL to make the following data structure
  \begin{verbatim}
    map< Enode *, vector< Enode * > > adj_list;
  \end{verbatim}
  that we declare as private attribute in \verb|SOSolver.h|, so that
  it will be accessible everywhere in the class
  \vfill
  When we receive a new constraint with \verb|assertLit|, we update
  \verb|adj_list| with 
  \begin{verbatim}
    adj_list[ from ].push_back( e );
  \end{verbatim}

\end{frame}

\begin{frame}[fragile]
  \frametitle{Phase 3 - Implementing the solver}
  \framesubtitle{Checking for a cycle}

  \scriptsize

  When \verb|check| is called we have to see if there is a cycle in the current graph
  \vfill
  We use a simple depth-first search using the following high-level recursive function:
  \vfill
  \begin{tabbing}
  sd \= sd \= sd \= asd \kill
  \> Input: a node ``from'' \\
  \> Output: $true$ iff a cycle containing ``from'' is found \\
  \> \\
  1 \> function findCycle( Enode * from ) \\
  \\
  2 \> \> if ( ``from was seen before'' ) return $true$ \\ 
  \\
  3 \> \> for each ``from $\leq$ y'' in adj\_list of from \\ 
  4 \> \> \> res = findCycle( y ) \\
  5 \> \> \> if ( res == $true$ ) return $true$ \\
  \\
  6 \> \> return $false$
  \end{tabbing}

\end{frame}

\begin{frame}[fragile]
  \frametitle{Phase 3 - Implementing the solver}
  \framesubtitle{Checking for a cycle}

  \scriptsize

  In SOSolver.C

  \begin{verbatim}
  ...
  bool SOSolver::findCycle( Enode * from )
  {
    if ( seen.find( from ) != seen.end( ) )
      return true;
    
    seen.insert( from );
    vector< Enode * > & adj_list_from = adj_list[ from ];
    
    for ( size_t i = 0 ; i < adj_list_from.size( ) ; i ++ )
    {
      const bool cycle_found = findCycle( adj_list_from[ i ]->get2nd( ) );
      if ( cycle_found ) 
        return true;
    }
    
    seen.erase( from );
    return false;
  }
  ...
  \end{verbatim}

\end{frame}

\begin{frame}[fragile]
  \frametitle{Phase 3 - Implementing the solver}
  \framesubtitle{Checking for a cycle}

  \scriptsize

  In SOSolver.C

  \begin{verbatim}
  bool SOSolver::check( bool complete )    
  { 
    for ( map< Enode *, vector< Enode * > >::iterator it = adj_list.begin( )
        ; it != adj_list.end( )
        ; it ++ )
    {
      seen.clear( );
    
      Enode * from = it->first;
      const bool cycle_found = findCycle( from ); 
    
      if ( cycle_found )
        return false;
    }
    
    return true;
  }
  \end{verbatim}

\end{frame}

\begin{frame}[fragile]
  \frametitle{Phase 3 - Implementing the solver}
  \framesubtitle{Checking for a cycle}

  In SOSolver.h

  \begin{verbatim}
  private:

    bool findCycle ( Enode * );
   
    map< Enode *, vector< Enode * > > adj_list;
    set< Enode * > seen;
  \end{verbatim}

  We can try to compile, and test it with files 
  so\_example\_1.smt2 and so\_example\_2.smt2

\end{frame}

\begin{frame}[fragile]
  \frametitle{Phase 3 - Implementing the solver}
  \framesubtitle{Computing conflicts}

  \scriptsize

  Conflicts can be computed by keeping track of
  the edges that form the last cycle detected 
  \vfill
  We add a \verb|map< Enode *, Enode * > parent_edge|
  map which we use to store the edge used to reach
  a node in the depth-serch traversal
  \vfill
  When during findCycle we discover an already visited
  node, we backward visit the parent\_edge relation
  to retrieve the cycle
  \vfill
  \begin{verbatim}
  void SOSolver::computeExplanation( Enode * from )
  {
    assert( explanation.empty( ) );
    Enode * x = from;
    do
    {
      x = parent_edge[ x ]->get1st( );
      explanation.push_back( parent_edge[ x ] );
    }
    while( x != from );
  }
  \end{verbatim}

\end{frame}

\begin{frame}[fragile]
  \frametitle{Phase 3 - Implementing the solver}
  \framesubtitle{Computing conflicts}

  \scriptsize

  Again in SOSolver.C

  \begin{verbatim}
  void SOSolver::findCycle( Enode * from )
  {
    if ( seen.find( from ) != seen.end( ) )
    {
      computeExplanation( from );
      return true;
    }
    ...
  }
  \end{verbatim}

  \vfill
  Again in SOSolver.h

  \begin{verbatim}
    map< Enode *, Enode * > parent_edge;
  \end{verbatim}

\end{frame}

\begin{frame}[fragile]
  \frametitle{Phase 3 - Implementing the solver}
  \framesubtitle{Incrementality - Backtrackability}

  Last thing we need is to handle incrementality/backtrackability
  \vfill
  For simplicity, solving will not be incremental nor backtrackable in this lecture
  \vfill
  However we still have to keep \verb|adj_list| updated with constraints received/dropped
  \vfill
  For this aim we introduce two more vectors in SOSolver.h

  \begin{verbatim}
  vector< Enode * > used_constr;
  vector< size_t >  backtrack_points;
  \end{verbatim}

  \verb|used_constr| keeps track of the order of constraints received,
  while \verb|backtrack_points| keeps track of the {\bf size} of 
  \verb|used_constr| when a new backtrack point is requested

\end{frame}

\begin{frame}[fragile]
  \frametitle{Phase 3 - Implementing the solver}
  \framesubtitle{Incrementality - Backtrackability}

  \scriptsize

  \begin{verbatim}
  void SOSolver::pushBacktrackPoint ( )
  {
    backtrack_points.push_back( used_constr.size( ) );
  }
  \end{verbatim}

  \begin{verbatim}
  void SOSolver::popBacktrackPoint ( )
  {
    assert( !backtrack_points.empty( ) );
    const size_t new_size = backtrack_points.back( );
    backtrack_points.pop_back( );
    
    while ( new_size < used_constr.size( ) )
    {
      Enode * e = used_constr.back( );
      Enode * from = e->get1st( );
      Enode * to   = e->get2nd( );
      assert( adj_list[ from ].back( ) == e );
      adj_list[ from ].pop_back( );
      used_constr.pop_back( );
    }
  }
  \end{verbatim}

\end{frame}
