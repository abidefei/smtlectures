\section{Lecture 3 - SAT-Solving}

\begin{frame}
  \frametitle{SAT-Solving}

  SAT-Solving is (the art of) finding the assignment satisfying all clauses
  \vfill
  The assignment evolves {\bf incrementally} by taking {\bf Decisions}, 
  starting from empty $\{\ \}$, but it can be {\bf backtracked} if found wrong
  \vfill
  The evolution of the assignment can be represented as a {\bf tree}
  \vfill
  \begin{tabular}{ccc}
    \begin{minipage}{.4\textwidth}
    $$
      \begin{array}{l}
      (\coltwoat{\neg a_1}{3-5} \vee a_3 \vee a_9) \\
      (\coloneat{\coltwoat{\neg a_2}{4}}{5} \vee \neg a_3 \vee a_4) \\
      (\coloneat{a_1}{3-5} \vee \coltwoat{\coloneat{a_2}{4}}{5}) \\
      (\neg a_4 \vee a_5 \vee a_{10})
      \end{array}
    $$
    \begin{overlayarea}{\textwidth}{.3cm}
      \only<1|handout:0>{$\{\ \}$}
      \only<2|handout:0>{$\{ \neg a_1 \}$}
      \only<3|handout:0>{$\{ \neg a_1, \neg a_2 \}$}
      \only<4|handout:0>{$\{ \neg a_1, a_2 \}$}
      \only<5>{$\{ \ldots \}$}
    \end{overlayarea}
    \end{minipage}
    & ~~~~ &
    \begin{minipage}{.4\textwidth}
    \begin{overlayarea}{\textwidth}{5cm}
      %\only<1|handout:0>{\scalebox{.7}{\input{sat_search_1.pdf_t}}}
      \only<1|handout:0>{\scalebox{.7}{\input{sat_search_1_1.pdf_t}}}
      \only<2|handout:0>{\scalebox{.7}{\input{sat_search_1_2.pdf_t}}}
      \only<3|handout:0>{\scalebox{.7}{\input{sat_search_1_3.pdf_t}}}
      \only<4|handout:0>{\scalebox{.7}{\input{sat_search_1_4.pdf_t}}}
      \only<5>{\scalebox{.7}{\input{sat_search_1_5.pdf_t}}}
    \end{overlayarea}
    \end{minipage}
  \end{tabular}

\end{frame}

\begin{frame}
  \frametitle{SAT-Solving}

  There are assignments which can be trivially driven towards the right direction.
  In the example below, given the current assignment, the third clause can be satisfied
  only by setting $a_2 \mapsto \top$ 
  \vfill
  \begin{tabular}{ccc}
    \begin{minipage}{.4\textwidth}
    $$
      \begin{array}{l}
      (\coltwoat{\neg a_1}{1-} \vee a_3 \vee a_9) \\
      (\coloneat{\neg a_2}{2-} \vee \neg a_3 \vee a_4) \\
      (\coloneat{a_1}{1-} \vee \coltwoat{a_2}{2-}) \\
      (\neg a_4 \vee a_5 \vee a_{10})
      \end{array}
    $$
    \begin{overlayarea}{\textwidth}{.3cm}
      \only<1|handout:0>{$\{ \neg a_1 \}$}
      \only<2->{$\{ \neg a_1, a_2 \}$}
    \end{overlayarea}
    \end{minipage}
    & ~~~~ &
    \begin{minipage}{.4\textwidth}
    \begin{overlayarea}{\textwidth}{4cm}
      \only<1|handout:0>{\scalebox{.7}{\input{sat_search_2_1.pdf_t}}}
      \only<2->{\scalebox{.7}{\input{sat_search_2_2.pdf_t}}}
    \end{overlayarea}
    \end{minipage}
  \end{tabular}
  \vfill
  \pause
  \pause
  This step is called {\bf Boolean Constraint Propagation} (BCP). It represents
  a {\bf forced} deduction. It triggers whenever all literals but one are assigned $\bot$
  $$
  (\colone{\neg a_1} \vee \colone{a_2} \vee \colone{\neg a_3} \vee \colone{a_4} \vee \neg a_5)
  $$

\end{frame}

\begin{frame}[fragile]
  \frametitle{SAT-Solving}

  {\bf Decision-level}: in an assignment, it is the number of decisions taken 
  \vfill
  Clearly, BCPs do not contribute to increase the decision level
  \vfill
  When necessary, we will indicate decision level on top of literals $\dec{a}{5}$ in the assignment 
  \vfill
  Example:
  \begin{center}
  \begin{tabular}{ccc}
    \begin{minipage}{.4\textwidth}
      \vfill
      $\{ \dec{a_{10}}{0}, \dec{\neg a_{1}}{1}, \dec{a_3}{2}, \dec{\neg a_4}{3}, \dec{a_5}{3} \}$
      \vfill
    \end{minipage}
    & &
    \begin{minipage}{.4\textwidth}
      \scalebox{.4}{\input{sat_search_4.pdf_t}}
    \end{minipage}
  \end{tabular}
  \end{center}
  
\end{frame}

\begin{frame}
  \frametitle{SAT-Solving}
  
  The process of finding the satisfying assignment is called {\bf search}, and
  in its basic version it evolves with {\bf Decisions}, {\bf BCPs}, and
  {\bf backtracking}

  \begin{center}
  \scalebox{.5}{\input{sat_search_3.pdf_t}}
  \end{center}

\end{frame}

\begin{frame}
  \frametitle{SAT-Solving}

  \scriptsize

  Consider the following scenario before and after BCP
  \vfill
  \begin{tabular}{ccc}
  \begin{minipage}{.4\textwidth}
  $$
  \begin{array}{l}
    \ldots \\
    (\colone{\neg a_{10}} \vee \colone{\neg a_1} \vee a_4) \\
    (\colone{a_3} \vee \colone{\neg a_1} \vee a_5) \\
    (\neg a_4 \vee a_6) \\
    (\neg a_5 \vee \neg a_6) \\
    \ldots \\
    \\
    \{ \dec{a_{10}}{0}, \dec{\neg a_3}{1}, \dec{a_7}{2}, \dec{\neg a_2}{3}, \dec{a_1}{4} \}
  \end{array}
  $$
  \end{minipage}
  & ~~~~ &
  \begin{minipage}{.5\textwidth}
  $$
  \begin{array}{l}
    \ldots \\
    (\colone{\neg a_{10}} \vee \colone{\neg a_1} \vee \coltwo{a_4}) \\
    (\colone{a_3} \vee \colone{\neg a_1} \vee \coltwo{a_5}) \\
    (\colone{\neg a_4} \vee \coltwo{a_6}) \\
    (\colone{\neg a_5} \vee \colone{\neg a_6}) \\
    \ldots \\
    \\
    \{ \dec{a_{10}}{0}, \dec{\neg a_3}{1}, \dec{a_7}{2}, \dec{\neg a_2}{3}, \dec{a_1}{4}, \dec{a_4}{4}, \dec{a_5}{4}, \dec{a_6}{4}\}
  \end{array}
  $$
  \end{minipage}
  \end{tabular}
  \vfill
  BCP leads to a conflict, but, what is the {\bf reason} for it ?
  Do $a_7$ and $a_2$ play any role ?
  \vfill
  Does it make sense to consider the assignments (which backtracking would produce) ?\\
  $\{ \dec{a_{10}}{0}, \dec{\neg a_3}{1}, \dec{\neg a_7}{2}, \dec{\neg a_2}{3}, \dec{a_1}{4} \}$ ---
  $\{ \dec{a_{10}}{0}, \dec{\neg a_3}{1}, \dec{a_7}{2}, \dec{\neg a_2}{3}, \dec{a_1}{4} \}$ ---
  $\{ \dec{a_{10}}{0}, \dec{\neg a_3}{1}, \dec{a_7}{2}, \dec{a_2}{3}, \dec{a_1}{4} \}$ \\
  \vfill
  No, because {\em whenever $a_{10}$, $\neg a_3$ is assigned, then $a_1$ must not be set to $\top$}.
  This translates to an additional clause, which can be {\em learnt}, i.e., it can be added to the formula 
  $$(\neg a_{10} \vee a_3 \vee \neg a_1)$$
   
\end{frame}

\begin{frame}
  \frametitle{SAT-Solving}

  \scriptsize

  In practice, we use {\bf resolution} steps to compute the learnt clause:
  \begin{itemize}
    \item we start from the clause with all literals to $\bot$ (conflicting clause)
    \item iteratively, we take the clause that propagated the last literal on the trail
	  and we apply resolution
    \item we stop when only one literal from the current decision level is left in the clause 
  \end{itemize}
  \vfill
  \begin{tabular}{ccc}
    \begin{minipage}{.4\textwidth}
      \scalebox{.4}{\input{ca_7.pdf_t}}
    \end{minipage}
    & ~~~~~~~~~~ &
    \begin{minipage}{.4\textwidth}
      \begin{tabular}{ccl}
	\hline
	Trail & dl & Reason \\
	\hline
	$a_{10}$   & 0 & $( a_{10} )$ \\
	$\neg a_3$ & 1 & Decision \\
	$a_7$      & 2 & Decision \\
	$\neg a_2$ & 3 & Decision \\
	$a_1$      & 4 & Decision \\
	$a_4$      & 4 & $(\neg a_{10} \vee \neg a_1 \vee a_4)$ \\
	$a_5$      & 4 & $(a_3 \vee \neg a_1 \vee a_5)$ \\
	$a_6$      & 4 & $(\neg a_4 \vee a_6)$ \\
	\hline
      \end{tabular}
      \medskip \\
      $\{ \dec{a_{10}}{0}, \dec{\neg a_3}{1}, \dec{a_7}{2}, \dec{\neg a_2}{3}, \dec{a_1}{4}, \dec{a_4}{4}, \dec{a_5}{4}, \dec{a_6}{4}\}$
    \end{minipage}
  \end{tabular}
  \vfill
  We say that $(\neg a_{10} \vee a_3 \vee \neg a_1)$ is the {\bf conflict clause} and it is the one we learn

\end{frame}

\begin{frame}
  \frametitle{Lecture 3 - Exercize 1}

  \scriptsize

  Consider the following clause set and assignment
  $$
  \begin{array}{l}
  (\neg a_1 \vee a_2) \\
  (\neg a_1 \vee a_3 \vee a_9) \\
  (\neg a_2 \vee \neg a_3 \vee a_4) \\
  (\neg a_4 \vee a_5 \vee a_{10}) \\
  (\neg a_4 \vee a_6 \vee a_{11}) \\
  (\neg a_5 \vee \neg a_6) \\
  (a_1 \vee a_7 \vee \neg a_{12}) \\
  (a_1 \vee a_8) \\
  (\neg a_7 \vee \neg a_8 \vee \neg a_{13}) \\
  \ldots \\ \\
  \{ \dec{\neg a_9}{1}, \dec{a_{12}}{2}, \dec{a_{13}}{2}, \dec{\neg a_{10}}{3}, \dec{\neg a_{11}}{3}, \ldots, \dec{a_1}{6}\}
  \end{array}
  $$
  \vfill
  \begin{itemize}
    \item Find the correct conflict clause
    \item Find the correct decision level to backtrack
  \end{itemize}

\end{frame}

\begin{frame}
  \frametitle{Lecture 3 - Exercize 1}

  \scriptsize

  \scalebox{.8}{
  \begin{tabular}{ccc}
    \begin{minipage}{.4\textwidth}
     $$
     \begin{array}{l}
     (\neg a_1 \vee a_2) \\
     (\neg a_1 \vee a_3 \vee a_9) \\
     (\neg a_2 \vee \neg a_3 \vee a_4) \\
     (\neg a_4 \vee a_5 \vee a_{10}) \\
     (\neg a_4 \vee a_6 \vee a_{11}) \\
     (\neg a_5 \vee \neg a_6) \\
     (a_1 \vee a_7 \vee \neg a_{12}) \\
     (a_1 \vee a_8) \\
     (\neg a_7 \vee \neg a_8 \vee \neg a_{13})
     \end{array}
     $$
    \end{minipage}
    & ~~~~~~ &
    \begin{minipage}{.4\textwidth}
      \begin{tabular}{ccl}
	\hline
	Trail & dl & Reason \\
	\hline
	$\neg a_{9}$  & 1 & Decision \\
	$a_{12}$      & 2 & Decision \\
	$a_{13}$      & 2 & (some clause) \\
	$\neg a_{10}$ & 3 & Decision \\
	$\neg a_{11}$ & 3 & (some clause) \\
	\ldots \\
	$a_1$         & 6 & Decision \\
	\hline 
	$a_2$         & 6 & $(\neg a_1 \vee a_2)$ \\
	$a_3$         & 6 & $(\neg a_1 \vee a_3 \vee a_9)$ \\
	$a_4$         & 6 & $(\neg a_2 \vee \neg a_3 \vee a_4)$ \\
	$a_5$         & 6 & $(\neg a_4 \vee a_5 \vee a_{10})$ \\
	$a_6$         & 6 & $(\neg a_4 \vee a_6 \vee a_{11})$ \\
	\hline
      \end{tabular}
      \bigskip \\
      $\{ \dec{\neg a_9}{1}, \dec{a_{12}}{2}, \dec{a_{13}}{2}, \dec{\neg a_{10}}{3}, \dec{\neg a_{11}}{3}, \ldots, \dec{a_1}{6} \}$
    \end{minipage}
  \end{tabular}}

  \vfill
  \pause

  \scalebox{.7}{
  \begin{minipage}{\textwidth}
    \begin{prooftree}
    \AxiomC{$(\neg a_5 \vee \neg a_6)$}
    \AxiomC{$(\neg a_4 \vee a_6 \vee a_{11})$}
    \BinaryInfC{$(\neg a_5 \vee \neg a_4 \vee a_{11})$}
    \AxiomC{$(\neg a_4 \vee a_5 \vee a_{10})$}
    \BinaryInfC{$(\neg a_4 \vee a_{10} \vee a_{11})$}
    \AxiomC{$(\neg a_2 \vee a_3 \vee a_4)$}
    \BinaryInfC{$(\neg a_2 \vee \neg a_3 \vee a_{10} \vee a_{11})$}
    \AxiomC{$(\neg a_2 \vee a_3 \vee a_9)$}
    \BinaryInfC{$(\neg a_2 \vee a_9 \vee a_{10} \vee a_{11})$}
    \AxiomC{$(\neg a_1 \vee a_2)$}
    \BinaryInfC{$(\neg a_1 \vee a_9 \vee a_{10} \vee a_{11})$}
    \end{prooftree}
  \end{minipage}
  }

  \vfill
  \pause
  Conflict clause: $(\neg a_1 \vee a_9 \vee a_{10} \vee a_{11})$ \pause \\
  Backtracking level: 3

\end{frame}
